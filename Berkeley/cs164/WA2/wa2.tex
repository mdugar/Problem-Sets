\documentclass[12t,letterpaper]{article}

\newenvironment{proof}{\noindent{\bf Proof:}}{\qed\bigskip}

\newtheorem{theorem}{Theorem}
\newtheorem{corollary}{Corollary}
\newtheorem{lemma}{Lemma} 
\newtheorem{claim}{Claim}
\newtheorem{fact}{Fact}
\newtheorem{definition}{Definition}
\newtheorem{assumption}{Assumption}
\newtheorem{observation}{Observation}
\newtheorem{example}{Example}
\newcommand{\qed}{\rule{7pt}{7pt}}

\newcommand{\assignment}[4]{
\thispagestyle{plain} 
\newpage
\setcounter{page}{1}
\noindent
\begin{center}
\framebox{ \vbox{ \hbox to 6.28in
        {\bf CS 164: Compilers\hfill #1}
\vspace{4mm}
\hbox to 6.28in
{\hspace{2.5in}\large\mbox{#2}}
\vspace{4mm}
\hbox to 6.28in
{{\it Handed Out: #3 \hfill Due: #4}}
}}
\end{center}
}

\newcommand{\solution}[3]{
\thispagestyle{plain} 
\newpage
\setcounter{page}{1}
\noindent
\begin{center}
\framebox{ \vbox{ \hbox to 6.28in
{\bf CS 164\hfill #3}
\vspace{4mm}
\hbox to 6.28in
{\hspace{2.5in}\large\mbox{#2}}
\vspace{4mm}
\hbox to 6.28in
{#1 \hfill}
}}
\end{center}
\markright{#1}
}

\newenvironment{algorithm}
{\begin{center}
\begin{tabular}{|l|}
\hline
\begin{minipage}{1in}
\begin{tabbing}
\quad\=\qquad\=\qquad\=\qquad\=\qquad\=\qquad\=\qquad\=\kill}
{\end{tabbing}
\end{minipage} \\
\hline
\end{tabular}
\end{center}}

\def\Comment#1{\textsf{\textsl{$\langle\!\langle$#1\/$\rangle\!\rangle$}}}


\usepackage{amsmath, verbatim, tikz, float, dsfont}


\usetikzlibrary{graphdrawing, graphdrawing.trees, arrows,automata}

\oddsidemargin 0in
\evensidemargin 0in
\textwidth 6.5in
\topmargin -0.5in
\textheight 9.0in
\newcommand{\norm}[1]{\left\lVert #1 \right\rVert}
\newcommand{\?}{\stackrel{?}{=}}

\begin{document}

\solution{Nikhil Unni (cs164-es), Section : Monday 3pm}{WA2}{Fall 2015}
\pagestyle{myheadings}


\begin{enumerate}
    \item Let L be the language consisting of all properly balanced forms of brackets in the alphabet ``\{, [, ], (, ), \}''. Write a context free grammar for the language L.\\\\

    $$S \rightarrow \{S\}$$
    $$S \rightarrow (S)$$
    $$S \rightarrow [S]$$
    $$S \rightarrow S S$$
    $$S \rightarrow \epsilon$$

    \item Consider the following grammar:
          $$S \rightarrow aSbS$$
          $$S \rightarrow bSaS$$
          $$S \rightarrow \epsilon$$

    \begin{enumerate}
      \item Give a one-sentence description of the language generated by this grammar.\\\\

      It is the language of all words $w \in (a | b)^*$, such that the number of a's is equal to the number of b's.\\

      \item Show that this grammar is ambiguous by giving a string that can be parsed in two different ways.\\

      For the word ``baba'':

      \begin{tikzpicture}[sibling distance=10em,
        every node/.style = {draw,circle, align=center}]

        \node {S}
          child { node [top color=orange] {b} }
          child { node {S} 
            child { node {$\epsilon$} }}
          child { node [top color=orange] {a} }
          child { node {S} [sibling distance=5em]
            child { node [top color=orange] {b} }
            child { node {$\epsilon$} }
            child { node [top color=orange] {a} }
            child { node {$\epsilon$} }};

      \end{tikzpicture}\\
      \begin{tikzpicture}[sibling distance=10em,
        every node/.style = {draw,circle, align=center}]

        \node {S}
          child[color=black] { node [top color=orange] {b} }
          child { node {S} [sibling distance=5em]
            child { node [top color=orange] {a} }
            child { node {$\epsilon$} }
            child { node [top color=orange] {b} }
            child { node {$\epsilon$} }}
          child { node [top color=orange] {a} }
          child { node {S}          
            child { node {$\epsilon$} }};

      \end{tikzpicture}

      \item Give an unambiguous grammar that accepts the same language as the grammar above.\\\\\\\\\\
      
      $$S \rightarrow T$$
      $$S \rightarrow \epsilon$$

      $$T \rightarrow Same$$
      $$T \rightarrow Diff$$

      $$Same \rightarrow a B_2 a$$
      $$Same \rightarrow b A_2 b$$

      $$Diff \rightarrow a S b$$
      $$Diff \rightarrow b S a$$

      $$A_2 \rightarrow a a T$$
      $$A_2 \rightarrow a T a$$
      $$A_2 \rightarrow T a a$$
      $$A_2 \rightarrow a a$$

      $$B_2 \rightarrow b b T$$
      $$B_2 \rightarrow b T b$$
      $$B_2 \rightarrow T b b$$
      $$B_2 \rightarrow T b b$$
      $$B_2 \rightarrow b b$$

      \item Give a grammar that accepts the same language as the grammar EXCEPT does not include the empty string. You are not allowed to use epsilon.

      $$S \rightarrow ab$$
      $$S \rightarrow ba$$
      $$S \rightarrow aSbS$$
      $$S \rightarrow bSaS$$
      $$S \rightarrow aSb$$
      $$S \rightarrow bSa$$
      $$S \rightarrow abS$$
      $$S \rightarrow baS$$
    \end{enumerate}

    \item Using the context-free grammar for Cool given in Section 11 of the Cool manula, draw a parse tree for the following expression.
    \begin{verbatim}
    case a of
        x : Int => a + 5 * 4;
        y : String => "Hi".concat(a);
        z : Bool => true;
    esac
    \end{verbatim}

    \begin{tikzpicture}[emph/.style={edge from parent/.style={very thin}}, scale=0.7]
    \begin{scope}%
    [tree layout,level distance=10mm,text depth=.1em,text height=.8em]
    
    \node {expr}
      child { node {case} }
      child { node {expr}
        child { node {ID}
          child { node {a} }}}
      child { node {of} }
      child { node {ID}
        child { node {x} }}
      child { node {$:$} }
      child { node {TYPE}
        child { node {Int} }}      
      child { node {$=>$} }
      child { node {expr}
        child { node {expr}
          child { node {ID}
            child { node {a} }}}
        child { node {+}}
        child { node {expr}
          child { node {expr}
            child { node {integer}
              child { node {5} }}}
          child { node {*}}
          child { node {expr}
            child { node {integer}
              child { node {4} }}}}}
      child { node {$;$} }
      child { node {ID}
        child { node {y} }}
      child { node {$:$} }
      child { node {TYPE}
        child { node {String} }} 
      child { node {$=>$} }
      child { node {expr} [sibling distance=1.7em]
        child { node {expr}
          child { node {string}
            child {node {``Hi''} }}}
        child { node {$.$} }
        child { node {ID}
          child { node {concat} }}
        child { node {$($}}
        child { node {expr}
          child { node {ID}
            child { node {a} }}}
        child { node {$)$}}}
      child { node {$;$} }
      child { node {ID}
        child { node {z} }}
      child { node {$:$} }
      child { node {TYPE}
        child { node {Bool} }}      
      child { node {$=>$} }
      child { node {expr}
        child { node {true} }}
      child { node {$;$} }
      child { node {esac} };
  \end{scope}
  \end{tikzpicture}\\

  \item What issues might a top down parser have with the grammar listed below? How would a bottom up parser avoid these issues?
  $$E \rightarrow E + T$$
  $$E \rightarrow T$$
  $$T \rightarrow T * F$$
  $$T \rightarrow F$$
  $$F \rightarrow (E)$$
  $$F \rightarrow \mathds{Z}$$

  The grammar is left recursive, which means that a top-down parser cannot parse it until we convert it to a right recursive grammar (and factor) that accepts the same language. A bottom-up parser avoids these issues, and can parse a left recursive grammar, because it builds up from the original string to the start symbol, so it doesn't matter if the first symbol in a transition is the original symbol itself (or, less confusingly, ``left recursive'').

\end{enumerate}

\end{document}
