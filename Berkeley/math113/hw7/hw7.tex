\documentclass[12t,letterpaper]{article}

\newenvironment{proof}{\noindent{\bf Proof:}}{\qed\bigskip}

\newtheorem{theorem}{Theorem}
\newtheorem{corollary}{Corollary}
\newtheorem{lemma}{Lemma} 
\newtheorem{claim}{Claim}
\newtheorem{fact}{Fact}
\newtheorem{definition}{Definition}
\newtheorem{assumption}{Assumption}
\newtheorem{observation}{Observation}
\newtheorem{example}{Example}
\newcommand{\qed}{\rule{7pt}{7pt}}

\newcommand{\assignment}[4]{
\thispagestyle{plain} 
\newpage
\setcounter{page}{1}
\noindent
\begin{center}
\framebox{ \vbox{ \hbox to 6.28in
{\bf Math 113: Abstract Algebra \hfill #1}
\vspace{4mm}
\hbox to 6.28in
{\hspace{2.5in}\large\mbox{#2}}
\vspace{4mm}
\hbox to 6.28in
{{\it Handed Out: #3 \hfill Due: #4}}
}}
\end{center}
}

\newcommand{\solution}[3]{
\thispagestyle{plain} 
\newpage
\setcounter{page}{1}
\noindent
\begin{center}
\framebox{ \vbox{ \hbox to 6.28in
{\bf Math 113 \hfill #3}
\vspace{4mm}
\hbox to 6.28in
{\hspace{2.5in}\large\mbox{#2}}
\vspace{4mm}
\hbox to 6.28in
{#1 \hfill}
}}
\end{center}
\markright{#1}
}

\newenvironment{algorithm}
{\begin{center}
\begin{tabular}{|l|}
\hline
\begin{minipage}{1in}
\begin{tabbing}
\quad\=\qquad\=\qquad\=\qquad\=\qquad\=\qquad\=\qquad\=\kill}
{\end{tabbing}
\end{minipage} \\
\hline
\end{tabular}
\end{center}}

\def\Comment#1{\textsf{\textsl{$\langle\!\langle$#1\/$\rangle\!\rangle$}}}


\usepackage{amsmath, amssymb, dsfont}

\newenvironment{amatrix}[1]{%
  \left(\begin{array}{@{}*{#1}{c}|c@{}}
}{%
  \end{array}\right)
}

\makeatletter
\renewcommand*\env@matrix[1][*\c@MaxMatrixCols c]{%
  \hskip -\arraycolsep
  \let\@ifnextchar\new@ifnextchar
  \array{#1}}
\makeatother

\newcommand{\?}{\stackrel{?}{=}}

\begin{document}

\solution{Nikhil Unni}{Homework \#7}{Fall 2015}
\pagestyle{myheadings}


\begin{enumerate}
\item Let F denote the additive group of continuous functions from $\mathds{R}$ to $\mathds{R}$.
      \begin{enumerate}
      \item Let $\phi : F \rightarrow \mathds{R}$ be defined by $f \rightarrow \int_{0}^{1} f(x)dx$ for all $f \in F$. Prove that $\phi$ is a group homomorphism.

      $$f,g \in F$$
      $$\phi(f+g) \? \phi(f) + \phi(g)$$
      $$\int_{0}^{1}(f+g)(x)dx \? \int_{0}^{1}f(x)dx+\int_{0}^{1}g(x)dx$$
      $$\int_{0}^{1}(f(x)+g(x))dx \? \int_{0}^{1}f(x)dx+\int_{0}^{1}g(x)dx$$
      $$\int_{0}^{1}f(x)dx+\int_{0}^{1}g(x)dx = \int_{0}^{1}f(x)dx+\int_{0}^{1}g(x)dx$$

      \item Find the kernel K of $\phi$.\\\\

      K is the set of all continuous functions where the integral from 0 to 1 equals 0. Algebraically, from the Fundamental Theorem of Calculus, this is the set of functions $f$ with antiderivative $F$, where $F(1) = F(0)$. Geometrically, this is the set of functions where the average value between 0 and 1 is 0 (like $sin(2\pi x)$, or $2x-1$).\\

      \item Describe the coset $x+K$ of the kernel, with both algebraic and geometric descriptions. Is there a ``nicer'' representative of this coset?\\\\

      It is the set of functions where the integral from 0 to 1 equals some constant, c. This means the average value from 0 to 1 is c. This can be represented as : $\{f_c | \int_{0}^{1}f_c dx = c, f_c \in F, c \in \mathds{R}\}$\\
      
      \item Note that the cosets of K are in bijection with $\mathds{R}$. Does Lagrange's Theorem apply here? Explain very briefly.\\\\

      Because the cosets of K form the quotient group of $G/ker(\phi)$, and since $\phi$ is a surjective function, by the Fundamental Homomorphism Theorem $\mathds{R}$ is isomorphic to $G/ker(\phi)$, meaning the mapping is bijective. Lagrange's Theorem does not help us here because it can only speak of cardinalities for infinite sets.\\
      
      \end{enumerate}

\item
        \begin{enumerate}
        \item Suppose G is a finite group of order m. Prove that $g^m=e$ for all $g \in G$.\\\\

        From Lagrange's Theorem, we know that the order of any $\langle g \rangle$ (let's call it $x$), for some $g \in G$ divides m. Meaning that $xy = m$, for some $y \in \mathds{Z}$. Then:
        $$g^m = g^{x+y} = (g^x)^y = e^y = e$$

        \item Suppose G is a finite group, and $N \trianglelefteq G$. If there are k cosets of N in G, prove that $g^k \in N$ for all $g \in G$.\\\\

        Because N is a normal subgroup, the set of cosets becomes the quotient group $G/N$ of order k, where $(aN)(bN)=(ab)N$ for the binary operation, and $eN=N$ is the identity. From part (a), we know that for any quotient group, q, $q^k=N$.\\
        By the definition of the binary operation, this means for any element $g$:
        $$(gN)^k=N$$
        $$(g^kN)=eN$$
        $$g^k=e$$
        \end{enumerate}

\item This problem will deal with the group $G = D_4 \times S_3$.
        \begin{enumerate}
        \item How many elements of each order do the groups $D_4$ and $S_4$ have? Using this info, determine how many elements of each order G has.
            \begin{center}
            For $D_4$:\\
            \begin{tabular}{c c}
               Order Number & Count\\
               \hline
               1 & 1\\
               2 & 5\\
               4 & 2\\\\\\
            \end{tabular}
            \end{center}
            
            \begin{center}
            For $S_3$:\\
            \begin{tabular}{c c}
               Order Number & Count\\
               \hline
               1 & 1\\
               2 & 3\\
               3 & 2\\\\\\
            \end{tabular}
            \end{center}

            \begin{center}
            For $G$:\\
            \begin{tabular}{c c}
               Order Number & Count\\
               \hline
               1 & 1\\
               2 & 29\\
               3 & 2\\
               4 & 2\\
               6 & 10\\
               12 & 4\\\\\\
            \end{tabular}
            \end{center}        

        \item Find all subgroups of G that are isomorphic to $Z_2 \times Z_2$. Be sure to explain why you have found all of them.\\\\

        All sets are of the form $\{(e,()), ds\}$ for all $d \in \{r^2, s, rs, r^2s, r^3s\}$, and $s \in \{(1,2), (2,3), (1,3)\}$. All of those elements in d and s are their own inverse in their respective group. So paired with the group's inverse, they become isomorphic to $Z_2$. 
        \end{enumerate}
\end{enumerate}
\end{document}
