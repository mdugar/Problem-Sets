\documentclass[12t,letterpaper]{article}

\newenvironment{proof}{\noindent{\bf Proof:}}{\qed\bigskip}

\newtheorem{theorem}{Theorem}
\newtheorem{corollary}{Corollary}
\newtheorem{lemma}{Lemma} 
\newtheorem{claim}{Claim}
\newtheorem{fact}{Fact}
\newtheorem{definition}{Definition}
\newtheorem{assumption}{Assumption}
\newtheorem{observation}{Observation}
\newtheorem{example}{Example}
\newcommand{\qed}{\rule{7pt}{7pt}}

\newcommand{\assignment}[4]{
\thispagestyle{plain} 
\newpage
\setcounter{page}{1}
\noindent
\begin{center}
\framebox{ \vbox{ \hbox to 6.28in
{\bf Math 113: Abstract Algebra \hfill #1}
\vspace{4mm}
\hbox to 6.28in
{\hspace{2.5in}\large\mbox{#2}}
\vspace{4mm}
\hbox to 6.28in
{{\it Handed Out: #3 \hfill Due: #4}}
}}
\end{center}
}

\newcommand{\solution}[3]{
\thispagestyle{plain} 
\newpage
\setcounter{page}{1}
\noindent
\begin{center}
\framebox{ \vbox{ \hbox to 6.28in
{\bf Math 113 \hfill #3}
\vspace{4mm}
\hbox to 6.28in
{\hspace{2.5in}\large\mbox{#2}}
\vspace{4mm}
\hbox to 6.28in
{#1 \hfill}
}}
\end{center}
\markright{#1}
}

\newenvironment{algorithm}
{\begin{center}
\begin{tabular}{|l|}
\hline
\begin{minipage}{1in}
\begin{tabbing}
\quad\=\qquad\=\qquad\=\qquad\=\qquad\=\qquad\=\qquad\=\kill}
{\end{tabbing}
\end{minipage} \\
\hline
\end{tabular}
\end{center}}

\def\Comment#1{\textsf{\textsl{$\langle\!\langle$#1\/$\rangle\!\rangle$}}}


\usepackage{amsmath, dsfont}

\newenvironment{amatrix}[1]{%
  \left(\begin{array}{@{}*{#1}{c}|c@{}}
}{%
  \end{array}\right)
}

\makeatletter
\renewcommand*\env@matrix[1][*\c@MaxMatrixCols c]{%
  \hskip -\arraycolsep
  \let\@ifnextchar\new@ifnextchar
  \array{#1}}
\makeatother

\newcommand{\?}{\stackrel{?}{=}}

\begin{document}

\solution{Nikhil Unni}{Homework \#5}{Fall 2015}
\pagestyle{myheadings}


\begin{enumerate}
\item Determine whether each of the following statements are true. If so give a proof, if not give a counterexample.
      \begin{enumerate}
      \item If G and H are groups suck that G has exactly $k$ different subgroups, and H has exactly $l$ different subgroups, then the group $G \times H$ has exactly $kl$ different subgroups.\\\\

      False. If G and H are both $Z_2$, then $G \times H$ is isomorphic to the Klein-4 group. And G and H both have 2 subgroups -- the trivial subgroup and themselves. But the Klein-4 group, $Z$, has 5 -- $\{e, \{e,a\}, \{e,b\}, \{e,c\}\}$ plus itself.\\

      \item Each dihedral group $D_n$, $n \geq 3$ is isomorphic to the direct product of a group of order $n$ and a group of order 2.\\\\

      False for $D_3$.\\\\
      Every group of order 2 has to be abelian because it can only contain $\{e, a = a^{-1}\}$\\
      And every group of order 3 also has to be abelian because it can only contain $\{e, a, a^{-1}\}$.
      And the product of two abelian groups is also abelian because the product is done component-wise (trivial proof), so the product of any group of 3 and 2 has to be abelian as well. But $D_3$ is not abelian, so the two cannot be isomorphic.\\
      \end{enumerate}

      \item In the group $D_6$, let H be the subgroup $\{e, r^2, r^4\}$
            \begin{enumerate}
            \item Do the left cosets and right cosets partition G the same way?\\\\

            Yes:
            $$eH = \{e, r^2, r^4\}$$
            $$rH = \{r, r^3, r^5\}$$
            $$sH = \{s, sr^2, sr^4\} = \{s, r^2s, r^4s\}$$
            $$rsH = \{rs, rsr^2, rsr^4\} = \{rs, r^3s, r^5s\}$$

            $$He = \{e, r^2, r^4\}$$
            $$Hr = \{r, r^3, r^5\}$$
            $$Hs = \{s, r^2s, r^4s\}$$
            $$Hrs = \{rs, r^3s, r^5s\}$$

            The two partitions are the same.\\

            \item Write out the group table for the group over cosets. What group is it isomorphic to?
            \begin{center}
            \begin{tabular}{c|c|c|c|c|}
             & eH & rH & sH & rsH\\
             \hline
             eH & eH & rH & sH & rsH\\
             rH & rH & eH & rsH & sH\\
             sH & sH & rsH & eH & rH\\
             rsH & rsH & sH & rH & eH\\
            \end{tabular}
          \end{center}

          It's isomorphic to the Klein-4 group, $V$.
          \end{enumerate}

      \item Try to find an example of each of the following. Show that your example meets the criterion, or prove no example is possible.
            \begin{enumerate}
            \item A subgroup $S_5$ which has order 10\\\\

            The dihedral group $D_5$ is a subgroup of $S_5$. It is the group generated by the set $\langle (1,2,3,4,5) , (2,5)(3,4) \rangle$. Geometrically, we can see that $(1,2,3,4,5)$ is a single rotation on the vertices, and $(2,5)(3,4)$ is the reflection over vertex 1. Since we know $D_5$ is just $\langle r , s \rangle$, and these are our two operations, the generated group is indeed $D_5$. And those elements (r and s) are of course members of $S_5$.\\

            \item An abelian subgroup of $S_5$ which has order 6.\\\\

            The cyclic group generated by $(1,2,3)(4,5)$ is isomorphic to $Z_6$, which is abelian. We can see the isomorphism by listing out the elements generated:\\
            $$(1,2,3)(4,5) , (1,3,2), (4,5), (1,2,3), (1,3,2)(4,5) , (), (1,2,3)(4,5), \ldots $$\\

            \item A nonabelian group which has subgroups of order 3, 6, and 7.\\\\

            The dihedral group $D_{42}$ has subgroups of order 3, 6, and 7. Ignoring $s$ generated elements in $D_{42}$, we can generate these subgroups just with rotations:\\\\
            The cyclic group generated by $\langle r^{14} \rangle$ has order 3, since it's just 3 distinct rotations.\\
            The cyclic group generated by $\langle r^7 \rangle$ has order 6, and the cyclic group generated by $\langle r^6 \rangle$ has order 7.\\

            \item A subgroup of $GL(2,\mathds{C})$ which has order 11\\\\

            Similar to part c, we can just generate this with rotations. The rotation matrix about the origin for some angle $\theta$ is:
            $$
            R = 
            \begin{pmatrix}[cc]
              cos(\theta) & -sin(\theta)\\
              sin(\theta) & cos(\theta)\\              
            \end{pmatrix}
            $$
            which is in $GL(2,\mathds{C})$.\\
            
            So the group generated by the rotation of $\frac{2\pi}{11}$ is a cyclic group that's isomorphic to $Z_{11}$, which has order 11. This is just:
            $$
            H = 
            \left \langle
            \begin{pmatrix}[cc]
              cos(\frac{2\pi}{11}) & -sin(\frac{2\pi}{11})\\
              sin(\frac{2\pi}{11}) & cos(\frac{2\pi}{11})\\              
            \end{pmatrix}
            \right \rangle
            $$
            

            \end{enumerate}
\end{enumerate}       
\end{document}
