\documentclass[12t,letterpaper]{article}

\newenvironment{proof}{\noindent{\bf Proof:}}{\qed\bigskip}

\newtheorem{theorem}{Theorem}
\newtheorem{corollary}{Corollary}
\newtheorem{lemma}{Lemma} 
\newtheorem{claim}{Claim}
\newtheorem{fact}{Fact}
\newtheorem{definition}{Definition}
\newtheorem{assumption}{Assumption}
\newtheorem{observation}{Observation}
\newtheorem{example}{Example}
\newcommand{\qed}{\rule{7pt}{7pt}}

\newcommand{\assignment}[4]{
\thispagestyle{plain} 
\newpage
\setcounter{page}{1}
\noindent
\begin{center}
\framebox{ \vbox{ \hbox to 6.28in
{\bf Math 113: Abstract Algebra \hfill #1}
\vspace{4mm}
\hbox to 6.28in
{\hspace{2.5in}\large\mbox{#2}}
\vspace{4mm}
\hbox to 6.28in
{{\it Handed Out: #3 \hfill Due: #4}}
}}
\end{center}
}

\newcommand{\solution}[3]{
\thispagestyle{plain} 
\newpage
\setcounter{page}{1}
\noindent
\begin{center}
\framebox{ \vbox{ \hbox to 6.28in
{\bf Math 113 \hfill #3}
\vspace{4mm}
\hbox to 6.28in
{\hspace{2.5in}\large\mbox{#2}}
\vspace{4mm}
\hbox to 6.28in
{#1 \hfill}
}}
\end{center}
\markright{#1}
}

\newenvironment{algorithm}
{\begin{center}
\begin{tabular}{|l|}
\hline
\begin{minipage}{1in}
\begin{tabbing}
\quad\=\qquad\=\qquad\=\qquad\=\qquad\=\qquad\=\qquad\=\kill}
{\end{tabbing}
\end{minipage} \\
\hline
\end{tabular}
\end{center}}

\def\Comment#1{\textsf{\textsl{$\langle\!\langle$#1\/$\rangle\!\rangle$}}}


\usepackage{amsmath, dsfont}

\newenvironment{amatrix}[1]{%
  \left(\begin{array}{@{}*{#1}{c}|c@{}}
}{%
  \end{array}\right)
}

\makeatletter
\renewcommand*\env@matrix[1][*\c@MaxMatrixCols c]{%
  \hskip -\arraycolsep
  \let\@ifnextchar\new@ifnextchar
  \array{#1}}
\makeatother

\begin{document}

\solution{Nikhil Unni}{Homework \#4}{Fall 2015}
\pagestyle{myheadings}


\begin{enumerate}
\item
This problem deals with subgroups of $GL(2,\mathds{R})$.
\begin{enumerate}
\item Prove that the set 
$$
  H = 
  \left \{
      \begin{pmatrix}[cc]
        a & b \\
        c & d \\
      \end{pmatrix}
      \in GL(2,\mathds{R}) : a,b,c,d \in \mathds{Z}
  \right \}
$$
is NOT a subgroup of $GL(2,\mathds{R})$.\\

We can disprove by example. Say we have two matrices from H:
$$
A =   \begin{pmatrix}[cc]
        1 & 0 \\
        1 & 1 \\
      \end{pmatrix},
B =   \begin{pmatrix}[cc]
        1 & 1 \\
        1 & 5 \\
      \end{pmatrix}
$$
which are both invertible (both $ad-bc$ are nonzero) and have integer values for every cell.

Subgroups have the condition where for all $a,b \in H$, $ab^{-1}$ must be in H as well. Given A and B above in H:
$$
AB^{-1} = \begin{pmatrix}[cc]
            \frac{5}{4} & \frac{-1}{4} \\
            1 & 0 \\
          \end{pmatrix}
$$
And because $AB^{-1}$ has noninteger cell values, it is not in H, therefore showing that H is not a subgroup of $GL(2,\mathds{R})$.\\\\
\item
  Find an infinite subset of the set H defined in part (a) which is a cyclic subgroup of $GL(2,\mathds{R})$.\\

  Let our cyclic subgroup generator be : 
$$
A = 
  \begin{pmatrix}[cc]
    1 & 1 \\
    0 & 1 \\
  \end{pmatrix}
$$
Where the inverse is:
$$
A^{-1} = 
  \begin{pmatrix}[cc]
    1 & -1 \\
    0 & 1 \\
  \end{pmatrix}
$$

I'll prove that any $A^n =   
  \begin{pmatrix}[cc]
    1 & n \\
    0 & 1 \\
  \end{pmatrix}$ for all $n \in \mathds{Z}$.

Base Case : for $n=0$ this is true, since this is just the identity matrix.
$$
A^0 = 
  \begin{pmatrix}[cc]
    1 & 0 \\
    0 & 1 \\
  \end{pmatrix}
$$
Recursive Case : Assume the inductive hypothesis (that the integer maps to ``b'' in the matrix):
$$
A^{n-1} = 
  \begin{pmatrix}[cc]
    1 & n-1 \\
    0 & 1 \\
  \end{pmatrix}
$$

Then by simple matrix multiplication we see:
$$
A^n = 
  \begin{pmatrix}[cc]
    1 & n \\
    0 & 1 \\
  \end{pmatrix}
$$
and 
$$
A^{n-2} = 
  \begin{pmatrix}[cc]
    1 & n-1 \\
    0 & 1 \\
  \end{pmatrix}
  \begin{pmatrix}[cc]
    1 & -1 \\
    0 & 1 \\
  \end{pmatrix}  
  = 
  \begin{pmatrix}[cc]
    1 & n-2 \\
    0 & 1 \\
  \end{pmatrix}  
$$
Which proves the inductive hypothesis for all $\mathds{Z}$ (since we can exponentiate forwards or backwards inductively from 0). A generates an infinite subset of H (all $A^n$ are invertible since $ad-bc = 1$, and all cells are in $\mathds{Z}$). Also, for all A,B in our generated subgroup:
$$
  \begin{pmatrix}[cc]
    1 & a \\
    0 & 1 \\
  \end{pmatrix}
  \begin{pmatrix}[cc]
    1 & -b \\
    0 & 1 \\
  \end{pmatrix}  
  = 
  \begin{pmatrix}[cc]
    1 & a-b \\
    0 & 1 \\
  \end{pmatrix}  
$$
And since integer addition and subtraction are closed, every $AB^{-1}$ is also in the subgroup, proving that it's a valid subgroup of H.\\\\

\item Find an infinite subset of H which is a noncyclic subgroup of $GL(2,\mathds{R})$.\\

Matrices of the form:
$$
  M = 
  \left \{
      \begin{pmatrix}[cc]
        a & b \\
        0 & d \\
      \end{pmatrix}
      \in GL(2,\mathds{R}) : a,b,d \in \mathds{Z}, ad = \pm 1
  \right \}
$$
This is a special case of H, as it constrains all 4 elements. But as we can see, it's closed under multiplication:
$$
      \begin{pmatrix}[cc]
        a & b \\
        0 & d \\
      \end{pmatrix}
      \begin{pmatrix}[cc]
        m & n \\
        0 & q \\
      \end{pmatrix}
  = 
      \begin{pmatrix}[cc]
        am & an+bq \\
        0 & dq \\
      \end{pmatrix}
$$
$$
(am)(dq) = (ad)(mq) = \pm 1
$$
We can also clearly see that the identity element is just the identity matrix, which is also of the same form. And because $ad-bc$ for all matrices in M is nonzero (because it is a subset of invertible matrices, all matrices in M must be invertible as well), every matrix in M has a multiplicative inverse:
$$
      \begin{pmatrix}[cc]
        a & b \\
        0 & d \\
      \end{pmatrix} ^{-1}
= 
     \frac{1}{\pm 1}
      \begin{pmatrix}[cc]
        d & -b \\
        0 & a \\
      \end{pmatrix} ^{-1}
$$
And we get matrix multiplication associativity for free. So it is a valid subgroup of $GL(2,\mathds{R})$. We can also prove by example that it's noncyclic. Since noncyclic subgroups are also commutative (because addition of exponents is commutative), we can just show that our subgroup is noncommutative:
$$
      \begin{pmatrix}[cc]
        -1 & 1 \\
         0 & 1 \\
      \end{pmatrix}
      \begin{pmatrix}[cc]
        1 & 1 \\
        0 & 1 \\
      \end{pmatrix}
  = 
      \begin{pmatrix}[cc]
        -1 & 0 \\
        0  & 1 \\
      \end{pmatrix}
$$
$$
      \begin{pmatrix}[cc]
        1 & 1 \\
        0 & 1 \\
      \end{pmatrix}
      \begin{pmatrix}[cc]
        -1 & 1 \\
         0 & 1 \\
      \end{pmatrix}
  = 
      \begin{pmatrix}[cc]
        -1 & 2 \\
        0  & 1 \\
      \end{pmatrix}
$$
So our subgroup is noncyclic.
\end{enumerate}
\item Here we will think about products of groups.
\begin{enumerate}
\item Prove that if G and H are groups, then $G \times H$, with a componentwise binary operation is a group.\\

First we show that $G \times H$ is closed under the componentwise binary operation. For two $a = (g_1,h_1),b = (g_2,h_2) \in G \times H$ : $ab = (g_1g_2,h_1h_2)$. Since G and H are groups, all $g_1g_2$ are in G, and all $h_1h_2$ are in H, so all $ab$ are in $G \times H$.\\

Next we show that every element has an identity by which it can multiply with to yield our identity, $(e_g,e_h)$, the tuple of G and H's identities:
$$ab = (g_1g_2,h_1h_2)$$
Then 
$$(ab)(ab)^{-1} = (g_1g_2,h_1h_2)((g_1g_2)^{-1},(h_1h_2)^{-1}) = (e_g,e_h)$$
Because G and H are groups, then all $g_1g_2$ and $h_1h_2$ multiplied by their repsective inverse yields the identity of the group. And the tuple of the identites (which is our new group's identity) is in $G \times H$.\\

Next, we can show that any element multiplied by our new identity yields itself:\\
$a = (g,h), e = (e_g,e_h)$, so $ae$ should be $a$.
$$(g,h)(e_g,e_h) = (ge_g,he_h) = (g,h)$$
Since any element in $G$ multiplied by $e_g$ yields itself, and any element in $H$ multiplied by $e_h$ yields itself, this formulation is correct.\\

Finally, we have to show associativity, that, for some $a=(g_1,h_1),b=(g_2,h_2),c=(g_3,h_3) \in G \times H$:
$$(ab)c = a(bc)$$
$$(ab)c = (g_1g_2,h_1h_2)(g_3,h_3) = ((g_1g_2)g_3, (h_1h_2)h_3)$$
$$a(bc) = (g_1,h_1)(g_2g_3,h_2h_3) = (g_1(g_2g_3),h_1(h_2h_3))$$
Because binary operations on G and H are both associative, these two end up being the same:
$$(g_1g_2g_3,h_1h_2h_3)$$
proving associativity for the componentwise binary operation on $G \times H$.\\

\item
 Consider the example $\mathds{C}^{*} \times \mathds{C}^{*}$. Find two subgroups of order 8 in $\mathds{C}^{*} \times \mathds{C}^{*}$ -- one which is cyclic and one which is not.\\

 1. Let our first cyclic subgroup be $U_1 \times U_8$, where the identity is $(1, 1)$, and the generator is $(1,e^{2pi(1/8)})$. The generator generates all 8 elements $(1, e^{2pi(n/8)})$, for $0 \geq n < 7$. And for any $a,b \in U_1 \times U_8$, 
$$ab^{-1} = (1, e^{2pi(n/8)})(1, e^{2pi(-m/8)}) = (1, e^{2pi(n-m/8)})$$
which is still in our subgroup.\\

 2. Let our second noncyclic subgroup be $U_2 \times U_4$. We can see that the order of the subgroup is 8, since the order of $U_2$ and $U_4$ are 2 and 4 respectively. It is also a valid subgroup of $\mathds{C}^{*} \times \mathds{C}^{*}$, since for any $a,b \in U_2 \times U_4$:
$$
ab^{-1} = (e^{2pi(n/2)}, e^{2pi(m/4)})(e^{2pi(-p/2)}, e^{2pi(-q/4)}) = (e^{2pi(n-p/2)}, e^{2pi(m-q/4)})
$$\\
which is still in our subgroup. We can show that nothing can generate all 8 elements by running through each element and its order:
$(e^{2pi(0/2)}, e^{2pi(0/4)})$, order 1\\
$(e^{2pi(0/2)}, e^{2pi(1/4)})$, order 4\\
$(e^{2pi(0/2)}, e^{2pi(2/4)})$, order 2\\
$(e^{2pi(0/2)}, e^{2pi(3/4)})$, order 4\\
$(e^{2pi(1/2)}, e^{2pi(0/4)})$, order 2\\
$(e^{2pi(1/2)}, e^{2pi(1/4)})$, order 4\\
$(e^{2pi(1/2)}, e^{2pi(2/4)})$, order 2\\
$(e^{2pi(1/2)}, e^{2pi(3/4)})$, order 4\\

Since none of the possible elements have order 8, none of them can serve as a generator, meaning that the group is noncyclic.

\end{enumerate}

\item This problem is about cyclic groups and generators.

\begin{enumerate}
\item Find two choices of n so that $Z_n$ has exactly 4 different generators. Justify your answer.

1. For n = 5, we can show that our only generators are $\bar{1}, \bar{4}, \bar{2}, \bar{3}$ exhaustively:\\

$\bar{0}$, order 1\\
$\bar{1}$, order 5\\
$\bar{2} \rightarrow \{\bar{2}, \bar{4}, \bar{1}, \bar{3}, \bar{0}\}$, order 5\\
$\bar{3} \rightarrow \{\bar{3}, \bar{1}, \bar{4}, \bar{2}, \bar{0}\}$, order 5\\
$\bar{4}$, order 5\\\\

2. For n = 8, we can show that our only generators are $\bar{1}, \bar{3}, \bar{5}, \bar{7}$ exhaustively:\\

$\bar{0}$, order 1\\
$\bar{1}$, order 8\\
$\bar{2}$, order 4\\
$\bar{3}, \rightarrow \{\bar{3}, \bar{6}, \bar{1}, \bar{4}, \bar{7}, \bar{2}, \bar{5}, \bar{0}\}$, order 8\\
$\bar{4}$, order 2\\
$\bar{5}, \rightarrow \{\bar{5}, \bar{2}, \bar{7}, \bar{4}, \bar{1}, \bar{6}, \bar{3}, \bar{0}\}$, order 8\\
$\bar{6}, \rightarrow \{\bar{6}, \bar{4}, \bar{2}, \bar{0}$, order 4\\
$\bar{7}$, order 8\\

\item Which of the following groups are cyclic groups?\\

1. $\mathds{Z}_2 \times \mathds{Z}_3$ is cyclic, and is generated by $(\bar{1},\bar{1})$.\\

  In order it will generate : 
$$\{(\bar{1},\bar{1}),(\bar{0},\bar{2}),(\bar{1},\bar{0}),(\bar{0},\bar{1}),(\bar{1},\bar{2}),(\bar{0},\bar{0})\}$$

2. $\mathds{Z}_2 \times \mathds{Z}_4$ is noncyclic. Because $\mathds{Z}_2 \times \mathds{Z}_4$ is isomorphic to $U_2 \times U_4$, it's the same problem as asking if $U_2 \times U_4$ is cyclic or not. And I already solved this exhaustively on problem 2b. Since $U_2 \times U_4$ is noncyclic, that means $\mathds{Z}_2 \times \mathds{Z}_4$ is noncyclic as well.

\end{enumerate}
\end{enumerate}

\end{document}
