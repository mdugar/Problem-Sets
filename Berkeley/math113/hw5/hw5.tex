\documentclass[12t,letterpaper]{article}

\newenvironment{proof}{\noindent{\bf Proof:}}{\qed\bigskip}

\newtheorem{theorem}{Theorem}
\newtheorem{corollary}{Corollary}
\newtheorem{lemma}{Lemma} 
\newtheorem{claim}{Claim}
\newtheorem{fact}{Fact}
\newtheorem{definition}{Definition}
\newtheorem{assumption}{Assumption}
\newtheorem{observation}{Observation}
\newtheorem{example}{Example}
\newcommand{\qed}{\rule{7pt}{7pt}}

\newcommand{\assignment}[4]{
\thispagestyle{plain} 
\newpage
\setcounter{page}{1}
\noindent
\begin{center}
\framebox{ \vbox{ \hbox to 6.28in
{\bf Math 113: Abstract Algebra \hfill #1}
\vspace{4mm}
\hbox to 6.28in
{\hspace{2.5in}\large\mbox{#2}}
\vspace{4mm}
\hbox to 6.28in
{{\it Handed Out: #3 \hfill Due: #4}}
}}
\end{center}
}

\newcommand{\solution}[3]{
\thispagestyle{plain} 
\newpage
\setcounter{page}{1}
\noindent
\begin{center}
\framebox{ \vbox{ \hbox to 6.28in
{\bf Math 113 \hfill #3}
\vspace{4mm}
\hbox to 6.28in
{\hspace{2.5in}\large\mbox{#2}}
\vspace{4mm}
\hbox to 6.28in
{#1 \hfill}
}}
\end{center}
\markright{#1}
}

\newenvironment{algorithm}
{\begin{center}
\begin{tabular}{|l|}
\hline
\begin{minipage}{1in}
\begin{tabbing}
\quad\=\qquad\=\qquad\=\qquad\=\qquad\=\qquad\=\qquad\=\kill}
{\end{tabbing}
\end{minipage} \\
\hline
\end{tabular}
\end{center}}

\def\Comment#1{\textsf{\textsl{$\langle\!\langle$#1\/$\rangle\!\rangle$}}}


\usepackage{amsmath, dsfont}

\newenvironment{amatrix}[1]{%
  \left(\begin{array}{@{}*{#1}{c}|c@{}}
}{%
  \end{array}\right)
}

\makeatletter
\renewcommand*\env@matrix[1][*\c@MaxMatrixCols c]{%
  \hskip -\arraycolsep
  \let\@ifnextchar\new@ifnextchar
  \array{#1}}
\makeatother

\newcommand{\?}{\stackrel{?}{=}}

\begin{document}

\solution{Nikhil Unni}{Homework \#5}{Fall 2015}
\pagestyle{myheadings}


\begin{enumerate}
\item This problem deals with $S_7$ and $U_{10}$.
  \begin{enumerate}
  \item Find an element of $S_7$ that has order 10. Call it x. List the elements of $G = \langle x \rangle$, the subgroup of $S_7$ generated by your element x.\\\\

  $x = (12345)(67)$
  
  $$x^1 = (12345)(67)$$
  $$x^2 = (13524)$$
  $$x^3 = (14253)(67)$$
  $$x^4 = (15432)$$
  $$x^5 = (67)$$
  $$x^6 = (12345)$$
  $$x^7 = (13524)(67)$$
  $$x^8 = (14253)$$
  $$x^9 = (15432)(67)$$
  $$x^{10} = e$$

  The only generators for my group G are $x^1, x^3, x^7$, and $x^9$. Because G is a cyclic group of order 10, it's isomorphic to $Z_{10}$, which only has 4 generators, so there cannot be any more.\\

  \item Determine the number of isomoprhisms $\phi : G \rightarrow U_{10}$.\\\\

  As we just demonstrated, $Z_{10}$, which is isomorphic to $U_{10}$ and $G$, only has 4 possible generators. As we showed in class with $U_6$, there are no other ways to possibly map two groups, so there are only 4 isomorphisms.
  \end{enumerate}

  \item This problem deals with dihedral groups and symmetry groups.

  \begin{enumerate}
  \item The groups $D_{12}$ and $S_4$ both have order 24. Prove that they are both nonabelian, but they are not isomorphic to each other.\\\\

  We can show $D_{12}$ is nonabelian by showing that two of the operations do not commute.
  $$s * (rs) \? s * (sr)$$
  Geometrically speaking, the left hand side is reflecting, rotating forwards, then reflecting. Intuitively, this is just rotating backwards. The right hand side can be reassociated:
  $$r^{-1} \? (ss)*r$$
  $$r^{-1} \neq r$$

  Similarly, elements in $S_4$ do not commute, and we can show it with an example:
  $$(134)(124) \? (124)(134)$$
  $$(12)(34) \neq (13)(24)$$

  However, the two groups are not isomorphic since their elements do not match. Intuitively, we know that elements of order 2 are their own inverse in their generated cyclic subgroup. This is only the case for elements with cycles of 2. The only 9 elements that have an order 2 are therefore : $(1,2), (1,3), (1,4), (2,3), (2,4), (3,4)$ and $(12)(34), (14)(24), (13)(24)$. So there are only 9 elements with order 2 in $S_4$. However in $D_{12}$, all of the elements of the form $r^ns$ are their own inverse. And there's an additional element $r^6$ which is its own inverse. So in total, $D_{12}$ has 13 elements of order 2, and $S_4$ only has 9, so they cannot be isomorphic.

  \item Does $D_{12}$ have a subgroup which is isomorphic to the Klein-4 group V? If so, find it and write out its group table.\\\\

  Yes, the subgroup of $\{e, r^6, s, r^6s\}$ is isomorphic to V.

  \begin{table}[h!]
  \centering
  \begin{tabular}{|c |c |c |c| c|} 
   \hline
     & e & $r^6$ & $s$ & $r^6s$ \\ [0.5ex] 
   \hline
   e & e & $r^6$ & $s$ & $r^6s$ \\ 
   $r^6$ & $r^6$ & e & $r^6s$ & s \\
   $s$ & s & $r^6s$ & e & $r^6$ \\
   $r^6s$ & $r^6s$ & s & $r^6$ & e\\
   \hline
  \end{tabular}
  \caption{Subgroup of $D_{12}$}
  \end{table}

  \begin{table}[h!]
  \centering
  \begin{tabular}{|c |c |c |c| c|} 
   \hline
     & 1 & a & b & c \\ [0.5ex] 
   \hline
   1 & 1 & a & b & c \\ 
   a & a & 1 & c & b \\
   b & b & c & 1 & a \\
   c & c & b & a & 1\\
   \hline
  \end{tabular}
  \caption{Cayley table of Klein-4 group}
  \end{table}

  \item Find a group of $D_{12}$ that is isomorphic to $S_3$.

  The group of $\{e, s, r^4, r^8, r^4s, r^8s\}$ is isomorphic to $S_3$.\\\\

  Let our mapping be of the generators : $\phi : s \rightarrow (12), r^4 \rightarrow (23)$. If we step through all the possible multiplications, we'll see that (12) and (23) generate 6 distinct elements, so the mapping is one-to-one and onto.\\
  (If s is not in the term $r^ns$, then it is a trivial mapping, where $\phi(r^ns^0) = \phi(r^n)$). Stepping through the other 3 terms ($r^4s, r^8s, s$), we see that the mapping is consistent.
  \end{enumerate}

  \item
        \begin{enumerate}
          \item In the group $D_6$ what is the subgroup L generated by $\{r^2, s\}$?\\\\

          Geometrically, we can see that this is just $D_3$. Everything about $D_6$ is the same, except now all the rotations are 2x as far, meaning that there are half the rotations. This means that there are 3 equally spaced rotations of $\frac{2\pi}{3}$, which is geometrically identical to $D_3$ (and $S_3$ for that matter). The reflections don't change anything geometrically.

          \item In $S_{10}$ what is the subgroup K generated by the two-element set $\{(18)(29), (37)(56)\}$? Is K a subgroup of $A_{10}$?\\\\

          The elements of K are $\{(), (18)(29), (37)(56), (18)(29)(37)(56)\}$

          If we try multiplying the elements, we see that both of the elements (call them a and b) squared yields the identity. Also, $a * (ab) = b$ and $b * (ab) = a$, which makes this group, element-for-element, isomorphic to the Klein-4 group, V.\\

          It is indeed a subgroup of $A_{10}$, since both elements are even permutations, and even permutations are closed under composition.
        \end{enumerate}
\end{enumerate}       
\end{document}
