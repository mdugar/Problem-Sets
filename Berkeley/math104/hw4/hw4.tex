\documentclass[12t,letterpaper]{article}

\newenvironment{proof}{\noindent{\bf Proof:}}{\qed\bigskip}

\newtheorem{theorem}{Theorem}
\newtheorem{corollary}{Corollary}
\newtheorem{lemma}{Lemma} 
\newtheorem{claim}{Claim}
\newtheorem{fact}{Fact}
\newtheorem{definition}{Definition}
\newtheorem{assumption}{Assumption}
\newtheorem{observation}{Observation}
\newtheorem{example}{Example}
\newcommand{\qed}{\rule{7pt}{7pt}}

\newcommand{\assignment}[4]{
\thispagestyle{plain} 
\newpage
\setcounter{page}{1}
\noindent
\begin{center}
\framebox{ \vbox{ \hbox to 6.28in
{\bf Math 104: Introduction to Analysis \hfill #1}
\vspace{4mm}
\hbox to 6.28in
{\hspace{2.5in}\large\mbox{#2}}
\vspace{4mm}
\hbox to 6.28in
{{\it Handed Out: #3 \hfill Due: #4}}
}}
\end{center}
}

\newcommand{\solution}[3]{
\thispagestyle{plain} 
\newpage
\setcounter{page}{1}
\noindent
\begin{center}
\framebox{ \vbox{ \hbox to 6.28in
{\bf Math 104 \hfill #3}
\vspace{4mm}
\hbox to 6.28in
{\hspace{2.5in}\large\mbox{#2}}
\vspace{4mm}
\hbox to 6.28in
{#1 \hfill}
}}
\end{center}
\markright{#1}
}

\newenvironment{algorithm}
{\begin{center}
\begin{tabular}{|l|}
\hline
\begin{minipage}{1in}
\begin{tabbing}
\quad\=\qquad\=\qquad\=\qquad\=\qquad\=\qquad\=\qquad\=\kill}
{\end{tabbing}
\end{minipage} \\
\hline
\end{tabular}
\end{center}}

\def\Comment#1{\textsf{\textsl{$\langle\!\langle$#1\/$\rangle\!\rangle$}}}


\usepackage{amsmath, dsfont}

\oddsidemargin 0in
\evensidemargin 0in
\textwidth 6.5in
\topmargin -0.5in
\textheight 9.0in
\newcommand{\norm}[1]{\left\lVert #1 \right\rVert}
\newcommand{\abs}[1]{\left\vert #1 \right\vert}
\newcommand{\?}{\stackrel{?}{=}}

\begin{document}

\solution{Nikhil Unni}{Assignment \#4}{Spring 2016}
\pagestyle{myheadings}

\begin{itemize}
  \item [9.1]
    Using the limit theorems 9.2-9.7, prove the following. Justify all steps.
    \begin{itemize}
      \item [(a)] $\lim_{} \frac{n+1}{n} = 1$\\\\

        From Theorem 9.4, the limit of products are the products of the limits:
        $$= \frac{\lim_{} n+1}{\lim_{} n}$$
        $$= \frac{n^{-1} \lim_{} n+1}{n^{-1} \lim_{} n}$$
        From Theorem 9.2, product of a limit and nonlimit is the limit of the product of the two:
        $$= \frac{\lim_{} 1 + \frac{1}{n} }{\lim_{} \frac{n}{n}}$$
        From Theorem 9.3, the limit of a sum is the sum of the limits:
        $$= \frac{\lim_{} 1 + \lim_{} \frac{1}{n}}{\lim_{} 1}$$
        From Theorem 9.7, we know that $\lim_{} \frac{1}{n^1} = 0$, since $1 > 0$.
        $$= \frac{1 + 0}{1} = 1$$

      \item [(b)] $\lim_{} \frac{3n+7}{6n-5} = \frac{1}{2}$\\\\
        Following the same steps as part a:\\

        From Theorem 9.4:
        $$= \frac{\lim_{} 3n+7}{\lim_{} 6n-5}$$
        $$= \frac{n^{-1} \lim_{} 3n+7}{n^{-1} \lim_{} 6n-5}$$
        From Theorem 9.2:
        $$= \frac{\lim_{} 3 + \frac{7}{n} }{\lim_{} 6 - \frac{5}{n} }$$
        From Theorem 9.3:
        $$= \frac{\lim_{} 3 + \lim_{} \frac{7}{n} }{\lim_{} 6 - \lim_{} \frac{5}{n}}$$
        From Theorem 9.7, we know that $\lim_{} \frac{1}{n^1} = 0$, since $1 > 0$. We also know from Theorem 9.2 that $\lim_{} 7 \frac{1}{n} = 7*0 = \lim_{} 5 \frac{1}{n} = 5*0 = 0$. So:
        $$= \frac{3 + 0}{6 - 0} = \frac{1}{2}$$

      \item [(c)] $\lim_{} \frac{17n^5 + 73n^4 - 18n^2 + 3}{23n^5 + 13n^3} = \frac{17}{23}$\\\\
        
        From Theorem 9.4:
        $$= \frac{\lim_{} 17n^5 + 73n^4 - 18n^2 + 3}{\lim_{} 23n^5 + 13n^3}$$
        $$= \frac{n^{-5} \lim_{} 17n^5 + 73n^4 - 18n^2 + 3}{n^{-5} \lim_{} 23n^5 + 13n^3}$$
        From Theorem 9.2:
        $$= \frac{\lim_{} 17 + \frac{73}{n} - \frac{18}{n^3} + \frac{3}{n^5}}{\lim_{} 23 + \frac{13}{n^2}}$$
        From Theorem 9.3:
        $$= \frac{\lim_{} 17 + \lim_{} \frac{73}{n} - \lim_{}\frac{18}{n^3} + \lim_{} \frac{3}{n^5}}{\lim_{} 23 + \lim_{} \frac{13}{n^2}}$$
        From Theorem 9.7, all of those $(\lim_{} \frac{1}{n^p}, p > 0)$ evaluate to 0. And we can factor out the constants via Theorem 9.2. Evaluating all of the constant limits, we get:
        $$= \frac{17 + 0 - 0 + 0}{23 + 0} = \frac{17}{23}$$
      
    \end{itemize}
  \item [9.3]
    Suppose $\lim_{} a_n = a, \lim_{} b_n = b$, and $s_n = \frac{a_n^3 + 4a_n}{b_n^2 + 1}$. Prove $\lim_{} s_n = \frac{a^3+4a}{b^2 + 1}$ carefully, using the limit theorems.\\\\

    From Theorem 9.4, we know we can separate the limits of the numerator and denominator:
    $$= \lim_{} s_n = \frac{\lim_{} a_n^3 + 4a_n}{\lim_{} b_n^2 + 1}$$
    From Theorem 9.3, we can separate the limit of a sum into the sum of limits:
    $$= \frac{\lim_{} a_n^3 + \lim_{} 4a_n}{\lim_{} b_n^2 + \lim_{} 1}$$
    From Theorem 9.4 again, we know we can separate the products:
    $$= \frac{\lim_{} a_n * \lim_{} a_n * \lim_{} a_n + \lim_{} 4 * \lim_{} a_n}{\lim_{} b_n * \lim_{} b_n + \lim_{} 1}$$
    Finally, we can start evaluating the limits. We know that $\lim a_n = a, \lim b_n = b$, and the limit of a constant is just the constant. So we finally get:
    $$= \frac{a * a * a + 4a}{b*b + 1}$$
    $$= \frac{a^3+4a}{b^2 + 1}$$\\

  \item [9.4]
    Let $s_1 = 1$ and for $n \geq 1$ let $s_{n+1} = \sqrt{s_n + 1}$.    
    \begin{itemize}
      \item [(a)] List the first four terms of $(s_n)$.

        $$s_1 = 1$$
        $$s_2 = \sqrt{2}$$
        $$s_3 = \sqrt{\sqrt{2} + 1}$$
        $$s_4 = \sqrt{\sqrt{\sqrt{2} + 1} + 1}$$
      \item [(b)] It turns out that $(s_n)$ converges. Assume this fact and prove the limit is $\frac{1}{2}(1 + \sqrt{5})$.\\\\

        If $(s_n)$ converges, then as $n \rightarrow \infty$, $s_{n+1} - s_n \rightarrow 0$. So for an arbitrarily high $n$, we can say that $s_{n+1} = s_n = x$, where $x$ is just a temporarily variable. Then, from our recurrence relation we have:
        $$x = \sqrt{x + 1}$$
        $$x^2 = x + 1$$
        $$x^2 - x - 1 = 0$$
        Solving with the quadratic formula, we get:
        $$x = \frac{1}{2}(1 \pm \sqrt{5})$$
        But we know that since $s_1 > 0$, and each following term is generated by a sqrt, which only generates positive values, we know that $\frac{1}{2}(1 - \sqrt{5})$ cannot possibly be a value, since it's negative ($5 > 1$, so $\sqrt{5} > \sqrt{1}$). So then, the value of $s_{n+1}$ as $n \rightarrow \infty$ approaches $\frac{1}{2}(1 + \sqrt{5})$.\\        
    \end{itemize}

  \item [9.12]
    Assume all $s_n \neq 0$ and that the limit $L = \lim_{} \abs{\frac{s_{n+1}}{s_n}}$ exists.
    \begin{itemize}
      \item [(a)] Show that if $L < 1$, then $\lim_{} s_n = 0$.\\\\
        
        If the ratio of subsequent terms is approaching some $\abs{L} < 1$, then for an aribtrarily high value of n, we know that $\abs{s_{x+k}} \rightarrow L^k \abs{s_x}$, where you can think of $s_x$ as engulfing all the $\frac{s_{n+1}}{s_n}$ values, combined with $s_1$, until the ratio has converged to the constant $L$. And since $\abs{L} < 1$, by Theorem 9.7, as $k$ continues to grow (i.e. as n starts to grow, since n = x+k, for some arbitrarily high x), $L^k \rightarrow 0$. So the entire $\abs{s_{x+k}} \rightarrow 0\abs{s_x} = 0$.
        
      \item [(b)] Show that if $L > 1$, then $\lim_{} \abs{s_n} = + \infty$.\\\\
        
        By Theorem 9.5, we know that:
        $$\lim_{} \abs{\frac{s_n}{s_{n+1}}} = \frac{1}{L}$$
        Since $L > 1$, $\frac{1}{L} < L$. And now, using $\frac{1}{L}$ as our new ``L'', by part a, we know $\lim_{} s_{n+1} = 0$. Since the limit of the reciprical of the fraction we are looking for evaluates to 0, we know from Theorem 9.10, that the original fraction evaluates to $+ \infty$.
    \end{itemize}
  \item [9.14]
    Let $p > 0$. Use Exercise 9.12 to show:
    $$\lim_{n \rightarrow \infty} \frac{a^n}{n^p} = 
    \begin{cases} 
        0 & \abs{a} \leq 1 \\
        +\infty & a > 1\\
        \text{does not exist } & a < -1
    \end{cases}
    $$\\\\

    Let $s_n = \frac{a^n}{n^p}$. Then:
    $$\frac{s_{n+1}}{s_n} = \frac{a^{n+1}n^p}{a^n (n+1)^p} = \frac{an^p}{(n+1)^p}$$
    $$=a(\frac{n}{n+1})^p$$
    Finally, evaluating $L = \lim_{} \frac{s_{n+1}}{s_n}$, we get:
    $$\lim_{} a(\frac{n}{n+1})^p$$
    From 9.2:
    $$= a\lim_{}(\frac{n}{n+1})^p$$
    From 9.4:
    $$= a(\Pi_{i=1}^p \lim_{} \frac{n}{n+1})$$
    From our proof in 9.1a (altering the steps only slightly, since it's the reciprocal), we know this just:
    $$L = a(\Pi_{i=1}^p 1) = a$$
    So if $\abs{a} \leq 1$, then $L \leq 1$. If $a = 1$ or $a = -1$, from theorems 9.7 and 9.2, $\lim_{n \to\ \infty} \frac{1}{n^p} = 0$, and $\lim_{n \to\ \infty} -1 \frac{1}{n^p} = 0$. If $\abs{a} < 1$, then $L < 1$, and from 9.12, we know that $\lim_{} s_n = \lim_{} \frac{a^n}{n^p} = 0$.\\

    Similarly, if $a > 1$, then $L > 1$, and by 9.12, we know that $\lim_{} s_n = + \infty$, since $s_n$ can only be a positive value.\\

    If $a < -1$, then $L < -1$, meaning that consequent values of $a_n$ will be alternating signs, and increasing in value with each iteration. Because the values of $s_n$ get larger and larger in magnitude, while also alternating sign, we know that the limit will not converge to a value, and it will also not converge to either $+ \infty$ or $- \infty$ since the sign flips with every iteration. Thus, the limit does not exist.\\
  \item [9.15]
    Show $\lim_{n \to\ \infty} \frac{a^n}{n!} = 0$ for all $a \in \mathds{R}$.\\\\

    Let $s_n = \frac{a^n}{n!}$. Then, $L = \frac{s_{n+1}}{s_n} = \frac{a^{n+1} n!}{a^n (n+1)!} = \frac{a}{n+1}$. Using the same theorems/reasoning as problem 9.1 (divide numerator/denominator by n, and evaluate), the limit of $L$ is equivalent to $\frac{\lim_{} \frac{a}{n} }{1 + \lim_{} \frac{1}{n}} = \frac{0}{1} = 0$.\\

    Since $L = 0$, using the proof from 9.12, we know that $\lim_{} \frac{a^n}{n!} = \lim_{} s_n = 0$.\\

  \item [9.17]
    Give a formal proof that $\lim_{} n^2 = +\infty$ using only Definition 9.8.\\\\

    For some arbitrary $M > 0$, if $n^2 > M$, then $n > \sqrt{M}$. So we can pick $N = \sqrt{M}$, for any value $M$. This way, for each $M > 0$, there is a number $N = \sqrt{M}$, so that both $n > N$ and $n^2 > M$. So by definition 9.8, $\lim_{} n^2 = +\infty$.\\

  \item [9.18]
    \begin{itemize}
      \item [(a)] Verify $1 + a + a^2 + \cdots + a^n = \frac{1 - a^{n+1}}{1-a}$ for $a \neq 1$.\\\\

        For $n = 0$:
        $$\frac{1 - a^{0+1}}{1-a}$$
        $$=\frac{1 - a}{1-a} = 1$$

        Assume that the equality holds for some $n$. Then we can show that it holds for $n+1$:
        $$1 + a + a^2 + \cdots + a^n + a^{n+1}$$
        $$= (\frac{1 - a^{n+1}}{1-a}) + a^{n+1}$$
        $$= \frac{1 - a^{n+1} + a^{n+1}(1-a)}{1-a}$$
        $$= \frac{1 - a^{n+2}}{1-a}$$
        Since the equality holds for $n=0$, and we can show that it holds for $n+1$ given it holds for $n$, by induction, the equality holds for all $n \geq 0$.\\
        
      \item [(b)] Find $\lim_{n \to\ \infty} (1 + a + a^2 + \cdots + a^n)$ for $\abs{a} < 1$.
        $$\lim_{n \to\ \infty} (1 + a + a^2 + \cdots + a^n)$$
        From part a:
        $$=\lim_{n \to\ \infty} \frac{1 - a^{n+1}}{1-a}$$
        From Theorem 9.7, we know that $\lim_{n \to\ \infty} a^n = 0$. So we get:
        $$=\lim_{n \to\ \infty} \frac{1 - 0a}{1-a} = \frac{1}{1-a}$$\\

      \item [(c)] Calculate $\lim_{n \to\ \infty} (1 + \frac{1}{3} + \cdots + \frac{1}{3^n})$.\\

        Let $a = \frac{1}{3}$. Then, the sequence becomes:
        $$=\lim_{n \to\ \infty} (1 + a + \cdots a^n)$$
        Since $\abs{\frac{1}{3}} < 1$, we've already solved this problem in part b. The result of the limit is:
        $$\frac{1}{1-a} = \frac{1}{1 - \frac{1}{3}} = \frac{3}{2}$$
      \item [(d)] What is $\lim_{n \to\ \infty} (1 + a + \cdots + a^n)$ for $a \geq 1$?\\\\

        From part a, we know this limit is equivalent to:
        $$\lim_{n \to\ \infty} \frac{1 - a^{n+1}}{1-a}$$
        From definition 9.8: for any $M > 0$, if $a^{n+1} > M$, then $n > log_a(M) - 1$. So for any $M$, we can pick $N = log_a(M)$. Since the log base $a \geq 1$, we know that $N$ is positive. Thus, $n > N$ implies $a^{n+1} > M$. So $\lim_{n \to\ \infty} a^{n+1} = + \infty$ for $a \geq 1$.\\

        Then, the limit of the numerator evaluates to $-\infty$, and the limit of the denominator evalutes to $1-a$. So from theorem 9.9, we know that the limit evaluates to $\infty$, since the denominator is a negative value (since $a > 1$).
    \end{itemize}
  \item [10.1]
    Which of the following sequences are increasing? decreasing? bounded?
    \begin{itemize}
      \item [(a)] $\frac{1}{n}$ : \textbf{Both bounded and decreasing}
      \item [(b)] $\frac{(-1)^n}{n^2}$ : \textbf{Bounded}
      \item [(c)] $n^5$ : \textbf{Increasing}
      \item [(d)] $\sin(\frac{n \pi}{7})$ : \textbf{Bounded}
      \item [(e)] $(-2)^n$ : \textbf{None of the choices}
      \item [(f)] $\frac{n}{3^n}$ : \textbf{Both bounded and decreasing}
    \end{itemize}
\end{itemize}

\end{document}
