\documentclass[12t,letterpaper]{article}

\usepackage[framemethod=default]{mdframed}

\newenvironment{proof}{\noindent{\bf Proof:}}{\qed\bigskip}

\newtheorem{theorem}{Theorem}
\newtheorem{corollary}{Corollary}
\newtheorem{lemma}{Lemma} 
\newtheorem{claim}{Claim}
\newtheorem{fact}{Fact}
\newtheorem{definition}{Definition}
\newtheorem{assumption}{Assumption}
\newtheorem{observation}{Observation}
\newtheorem{example}{Example}
\newcommand{\qed}{\rule{7pt}{7pt}}

\newcommand{\assignment}[4]{
\thispagestyle{plain} 
\newpage
\setcounter{page}{1}
\noindent
\begin{center}
\framebox{ \vbox{ \hbox to 6.28in
{\bf Phil 140b: Intermediate Logic \hfill #1}
\vspace{4mm}
\hbox to 6.28in
{\hspace{2.5in}\large\mbox{#2}}
\vspace{4mm}
\hbox to 6.28in
{{\it Handed Out: #3 \hfill Due: #4}}
}}
\end{center}
}

\newcommand{\solution}[3]{
\thispagestyle{plain} 
\newpage
\setcounter{page}{1}
\noindent
\begin{center}
\framebox{ \vbox{ \hbox to 6.28in
{\bf Phil 140b \hfill #3}
\vspace{4mm}
\hbox to 6.28in
{\hspace{2.5in}\large\mbox{#2}}
\vspace{4mm}
\hbox to 6.28in
{#1 \hfill}
}}
\end{center}
\markright{#1}
}

\newcommand\tarski{\mathrel{|}\!=}

\newenvironment{algorithm}
{\begin{center}
\begin{tabular}{|l|}
\hline
\begin{minipage}{1in}
\begin{tabbing}
\quad\=\qquad\=\qquad\=\qquad\=\qquad\=\qquad\=\qquad\=\kill}
{\end{tabbing}
\end{minipage} \\
\hline
\end{tabular}
\end{center}}

\newenvironment{question}
{\begin{quote}\itshape}
{\end{quote}}

\def\Comment#1{\textsf{\textsl{$\langle\!\langle$#1\/$\rangle\!\rangle$}}}


\usepackage{amsmath, dsfont, mathtools, verbatim, tikz, float, textcomp, mathrsfs, amssymb}

\usetikzlibrary{arrows,automata}

\oddsidemargin 0in
\evensidemargin 0in
\textwidth 6.5in
\topmargin -0.5in
\textheight 9.0in
\newcommand{\norm}[1]{\left\lVert #1 \right\rVert}
\newcommand{\abs}[1]{\left\vert #1 \right\vert}
\newcommand{\?}{\stackrel{?}{=}}
\newcommand{\s}[1]{\overbrace{0''^{\cdots}}^{\text{#1 times}}}
\newcommand{\brak}[1]{\ulcorner #1 \urcorner}
\newcommand{\Diag}[1]{\text{Diag}(#1)}
\DeclarePairedDelimiter{\ceil}{\lceil}{\rceil}




\begin{document}

\solution{Nikhil Unni}{Assignment \#5}{Spring 2016}
\pagestyle{myheadings}

\begin{enumerate}
  \item
    \begin{question}
      Consider the theory Q given in class (this is the theory called R in section 16.4 of the book). Show by metatheoretical induction that if $i*j = k$, then $\tarski_Q \textbf{i}*\textbf{j} = \textbf{k}$.
    \end{question}
    First, let's prove that $i+j=k \implies \s{i} + \s{j} = \s{i+j}$. We'll do this through induction:\\

    Base Case: If $j=0$, then $i+j=k=i$, and by (Q3), we know that:
    $$\textbf{i} + \textbf{0} = \textbf{i}$$
    $$\textbf{i} + \textbf{j} = \textbf{k}$$

    Inductive Case: Assume for all $n < j$ that $i+n = k \implies \s{i} + \s{n} = \s{k}$. Then from (Q4) we have:
    $$\s{i} + (\s{j-1})' = (\s{i} + \s{j-1})'$$
    Using our inductive hypothesis:
    $$= (\s{i+j-1})' = \s{i+j}$$\\

    Now we want to show that $\s{i} * \s{j} = \s{k}$. And we'll do this through induction:\\




    Base Case : If $j=0$, then $i*j = k = 0$, and by (Q5), we know that:
    $$\textbf{i} * \textbf{0} = \textbf{0}$$
    $$\textbf{i} * \textbf{j} = \textbf{k}$$

    Inductive Case : Assume for all $n < j$ that $i*n = k \implies \s{i} * \s{n} = \s{k}$. Then from (Q6) we have:
    $$\s{i} * (\s{j-1})' = (\s{i} * \s{j-1}) + \s{i}$$
    Using our inductive hypothesis:
    $$= (\s{i(j-1)}) + \s{i} = \s{ij - i} + \s{i}$$
    And using our last proof:
    $$= \s{ij - i + i} = \s{ij}$$

    \pagebreak
  \item
    \begin{question}
      A formula $B(y)$ is called a truth-predicate for T if for any sentence G of the language T, $\tarski_T G \iff B(\brak{G})$. Show that if T is a consistent theory in which diag is representable, then there is no truth-predicate for T.
    \end{question}

    Assume there is such a truth-predicate, $B(y)$ in our consistent theory T. Let's construct a few preliminary sentences:
    $$A(x) \iff \exists y (\Diag{x,y} \land \neg B(y))$$
    $$a = \brak{A(x)}$$
    (This is using Diag as the representation of diag in T, as per the book's convention.)
    $$G \iff \exists x (x = \textbf{a} \land A(x))$$
    $$g = \brak{G}$$
    By construction $G$ is equivalent to:
    $$\iff \exists x (x = \textbf{a} \land \exists y (\Diag{x,y} \land \neg B(y))) \iff \exists y (\Diag{\textbf{a},y} \land \neg B(y))$$
    So then we have:
    $$\tarski_T G \iff \exists y (\Diag{\textbf{a},y} \land \neg B(y))$$
    And since $G$ is the diagonalization of $A(x)$ we also have:
    $$\tarski_T \forall y (\Diag{\textbf{a},y} \iff y = \textbf{g})$$
    Combining the last two formulas we get:
    $$\tarski_T G \iff \exists y (y = \textbf{g} \land \neg B(y))$$
    Or:
    $$\tarski_T G \iff \neg B(\textbf{g}) \iff \neg B(\brak{G})$$
    But this is a contradiction with the original definition of $B(y)$, meaning that no such truth-predicate can exist.

  \item
    \begin{question}
      A set $S$ of natural numbers is called recursively enumerable (r.e.) if there is a two-place recursive relation R such that $S = \{ x : \exists y R x y \}$. Show that for any set $S$, $S$ is recursive iff both $S$ and its complement ($\mathds{N}-S$) are recursively enumerable.
    \end{question}

    If $S$ is recursive:\\
    then it must have some recursive characteristic function $f_S$ s.t. $s \in S \iff f_S(s) = 1$ and $s \not \in S \iff f_S(s) = 0$. Then we can easily define our relations:
    $$R_S x y \iff (f_S(x)=y) \land (y=1)$$
    $$R_{\mathds{N}-S} x y \iff (f_S(x)=y) \land (y=0)$$\\

    If $S$ and $\mathds{N} - S$ are r.e.:\\
    then there must exist relations $R_S$ and $R_{\mathds{N}-S}$ as described above. We can construct a recursive characteristic function to show $S$ is recursive:
    $$f_S(x) = f'_S(x,0)$$
    $$f'_S(x,y) =
    \begin{cases}
      1 & R_S x y \\
      0 & R_{\mathds{N}-S} x y\\
      f'_S(x,y+1) & \text{else}
    \end{cases}
    $$
    $f'_S(x,0)$ (and therefore $f_S$) is guaranteed to halt in a finite amount of time because for a given $x$ either $R_S$ or $R_{\mathds{N}-S}$ is guaranteed to be true for some $y$. This means we have an effective procedure, and so $f_S$ is a valid recursive characteristic function.




  \item
    \begin{question}
      Show that all r.e. sets are definable in arithmetic (i.e. the theory consisting of $L(Q)$ that are true in $\mathds{N}$).
    \end{question}

    All r.e. sets have some two-place recursive relation R such that $S = \{x : \exists y R x y \}$. Furthermore, it was shown in the book (not the chapter PDFs) in Theorem 16.16 that every recursive relation is representable in Q. Say that $\phi(x,y,w)$ represents the recursive relation in Q. Then we can arithmetically define r.e. sets as:
    $$F(x,y) \iff \phi(x,y,1)$$
\end{enumerate}

\end{document}