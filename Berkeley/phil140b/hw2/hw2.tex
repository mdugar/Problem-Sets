\documentclass[12t,letterpaper]{article}

\usepackage[framemethod=default]{mdframed}

\newenvironment{proof}{\noindent{\bf Proof:}}{\qed\bigskip}

\newtheorem{theorem}{Theorem}
\newtheorem{corollary}{Corollary}
\newtheorem{lemma}{Lemma} 
\newtheorem{claim}{Claim}
\newtheorem{fact}{Fact}
\newtheorem{definition}{Definition}
\newtheorem{assumption}{Assumption}
\newtheorem{observation}{Observation}
\newtheorem{example}{Example}
\newcommand{\qed}{\rule{7pt}{7pt}}

\newcommand{\assignment}[4]{
\thispagestyle{plain} 
\newpage
\setcounter{page}{1}
\noindent
\begin{center}
\framebox{ \vbox{ \hbox to 6.28in
{\bf Phil 140b: Intermediate Logic \hfill #1}
\vspace{4mm}
\hbox to 6.28in
{\hspace{2.5in}\large\mbox{#2}}
\vspace{4mm}
\hbox to 6.28in
{{\it Handed Out: #3 \hfill Due: #4}}
}}
\end{center}
}

\newcommand{\solution}[3]{
\thispagestyle{plain} 
\newpage
\setcounter{page}{1}
\noindent
\begin{center}
\framebox{ \vbox{ \hbox to 6.28in
{\bf Phil 140b \hfill #3}
\vspace{4mm}
\hbox to 6.28in
{\hspace{2.5in}\large\mbox{#2}}
\vspace{4mm}
\hbox to 6.28in
{#1 \hfill}
}}
\end{center}
\markright{#1}
}

\newcommand\tarski{\mathrel{|}\!=}

\newenvironment{algorithm}
{\begin{center}
\begin{tabular}{|l|}
\hline
\begin{minipage}{1in}
\begin{tabbing}
\quad\=\qquad\=\qquad\=\qquad\=\qquad\=\qquad\=\qquad\=\kill}
{\end{tabbing}
\end{minipage} \\
\hline
\end{tabular}
\end{center}}

\newenvironment{question}
{\begin{quote}\itshape}
{\end{quote}}

\def\Comment#1{\textsf{\textsl{$\langle\!\langle$#1\/$\rangle\!\rangle$}}}


\usepackage{amsmath, dsfont, mathtools, verbatim, tikz, float, textcomp, mathrsfs}

\usetikzlibrary{arrows,automata}

\oddsidemargin 0in
\evensidemargin 0in
\textwidth 6.5in
\topmargin -0.5in
\textheight 9.0in
\newcommand{\norm}[1]{\left\lVert #1 \right\rVert}
\newcommand{\?}{\stackrel{?}{=}}
\DeclarePairedDelimiter{\ceil}{\lceil}{\rceil}

\begin{document}

\solution{Nikhil Unni}{Assignment \#2}{Spring 2016}
\pagestyle{myheadings}

\begin{enumerate}
  \item
    \begin{question}
      Let $L^* = $ \{0, ', +, ., $<$ \}. Taking for granted the inductive definition of the terms of $L^*$ provided in class, define the atomic formulas of $L^*$ as follows:\\
      Atomic formulas: if $t_1, t_2$ are terms of $L^*$ then $=(t_1,t_2)$ and $<(t_1,t_2)$ are atomic formulas. Now define inductively the class of well formed formulas of $L^*$ as follows:\\
      Basis: all atomic formulas are well formed formulas;\\
      Inductive clause: if A and B are well formed formulas, so are $(A \land B), \neg A$, and $\forall xA$;\\
      External clause: nothing else is a well formed formula
    \end{question}

    \begin{enumerate}
      \item [a.]
        \begin{question}
          Show by induction on the construction of the set of well formed formulas that all well formed formulas have the same number of left and right parentheses. [You can assume in your proof that all terms of $L^*$ have the same number of left and right parentheses ]\\
        \end{question}

        Base Case: all atomic formulas are of the form: $=(t_1,t_2)$ or $<(t_1,t_2)$. Since it is assumed that all terms of $L^*$ have balanced parentheses, all atomic formulas must be balanced as well, since they just add a single left paren and right paren to the expression.\\

        Inductive Case: Assuming that well-formed formulas $A$ and $B$ have balanced parentheses, then:\\
        \begin{itemize}
          \item $(A \land B)$ must have balanced parentheses, since it adds a single left and right paren to the already balanced expression (and the ampersand doesn't change anything).
          \item $\neg A$ must have balanced parentheses, since it doesn't add any left or right parentheses to the expression.
          \item $\forall x A$ must have balanced parentheses as well, since it doesn't add any left or right parentheses.
        \end{itemize}

         Since well-formed formulas are inductively defined this way, by induction we have shown that all possible well-formed expressions have balanced parentheses.
        
      \item [b.]
        \begin{question}
          Following the outline of the proof we did in class for terms of $L^*$, define a numerical measure of complexity for the well formed formulas $(f(w) = n)$ and prove by induction on the natural numbers that ``for all $n$, for all well formed formulas $w$, if $\text{comp}(w) = n$, then $w$ has the same number of left and right parentheses''.\\
        \end{question}

        We define a ``complexity'' function $f: \text{well formed formulas} \rightarrow \mathds{N}$, which just maps a well formed formula to the maximum recursive depth to atomic formulas. Concretely:
        $$f(=(t_1,t_2)) = f(<(t_1,t_2)) = 0$$ 
        $$f((A \land B)) = \max(f(A), f(B)) + 1$$
        $$f(\neg A) = f(A) + 1$$
        $$f(\forall x A) = f(A) + 1$$

        Assuming all well-formed formulas are finitely long, they must have a finite complexity, and so for all $x \in \text{well formed formulas}$, $f(x) = n \in \mathds{N}$.\\

        Base Case: All formulas of complexity 0 are atomic formulas. Since it's assumed that all terms in $L^*$ have equal parentheses, we know all atomic formulas have balanced left and right parentheses, since they just add a single left and a single right. In other words, if $t_1$ has $n$ of both ($\#_L(t_1) = \#_R(t_1) = n$), and $t_2$ has $m$ of both ($\#_L(t_2) = \#_R(t_2) = m$), then:
        $$\#_L(=(t_1,t_2)) = \#_L(<(t_1,t_2)) = n + m + 1 = \#_R(=(t_1,t_2)) = \#_R(<(t_1,t_2))$$

        Inductive Case: Assuming that all formulas of complexity $\leq n$ have balanced parentheses, then we can show that all formulas of complexity $n+1$ have balanced parentheses. Say we have two well-formed formulas of complexity $<= n$ : A and B, where $\#_L(A) = \#_R(A) = x$, and $\#_L(B) = \#_R(B) = y$. Then:
        $$\#_L((A \land B)) = 1 + x + y = \#_R((A \land B))$$
        $$\#_L(\neg A) = x = \#_R(\neg A)$$
        $$\#_L(\forall x A) = x = \#_R(\forall x A)$$

        So by induction, all well-formed formulas of complexity $n \geq 0$ have balanced parentheses, and so all well-formed formulas must have balanced parentheses

    \end{enumerate}


  \item 
    \begin{question}
      Show that:
    \end{question}

    \begin{enumerate}
      \item
        \begin{question}
          If the sentence E is implied by the set of sentences $\Delta$ and every sentence $D$ in $\Delta$ is implied by the set of sentences $\Gamma$, then $E$ is implied by $\Gamma$.
        \end{question}

        Without loss of generation, let $\gamma$ be an arbitrary interpretation of the language. By definition, we know:
        $$(\forall D \in \Delta (\gamma \tarski D)) \implies (\gamma \tarski E)$$
        $$\forall D \in \Delta ((\forall G \in \Gamma (\gamma \tarski G) \implies (\gamma \tarski D))$$

        So then we know that:
        $$\forall D \in \Delta, \forall G \in \Gamma ((\gamma \tarski G) \implies (\gamma \tarski D) \implies (\gamma \tarski E))$$
        which means that:
        $$(\forall G \in \Gamma (\gamma \tarski G)) \implies (\gamma \tarski E)$$
        Meaning that $\Gamma \tarski E$.\\
        

      \item 
        \begin{question}
          \item If the sentence $E$ is implied by the set of sentences $\Gamma \cup \Delta$ and every sentence $D$ in $\Delta$ is implied by the set of sentences $\Gamma$, then $E$ is implied by $\Gamma$.
        \end{question}


        For future shorthand, say that for interpretation $\gamma$ and set of sentences $\chi$, $\gamma \tarski \chi$ means $\forall x \in \chi (\gamma \tarski x)$.\\

        Without loss of generlization, let $\gamma$ be an arbitrary interpretation of the language. We know: 
        $$((\gamma \tarski \Gamma) \land (\gamma \tarski \Delta)) \implies (\gamma \tarski E)$$
        $$\forall D \in \Delta ((\gamma \tarski \Gamma) \implies (\gamma \tarski D))$$

        So then we know:
        $$(\gamma \tarski \Gamma) \implies (\gamma \tarski \Gamma) \land (\forall D \in \Delta (\gamma \tarski D)) \implies (\gamma \tarski \Gamma) \land (\gamma \tarski \Delta) \implies (\gamma \tarski E)$$
        
    \end{enumerate}

  \item 
    \begin{question}
      Let $L = $ \{0, ', +, * \}. Give an interpretation of $L$ with a finite domain that makes the following sentences true:
      $$\neg \forall x \forall y (x * y = y * x)$$
      $$\neg \forall x \forall y (x+y = y+x)$$
    \end{question}

    This is equivalent to the two statements:
    $$\exists x \exists y (x * y \neq y * x)$$
    $$\exists x \exists y (x + y \neq y + x)$$

    So let's define an interpretation $\gamma$ with finite set-theoretic domain $\{0,1\}$, where $0^{\gamma} = 0$, and $(0')^{\gamma} = 1$, $(0'')^{\gamma} = 0$, etc. And we define our operators as:
    $$*^{\gamma} = \{(0,0,0), (0,1,1), (1,0,0), (1,1,1)$$
    $$+^{\gamma} = \{(0,0,1), (0,1,0), (1,0,1), (1,1,0)\}$$

    Then, we have $x = 0$ and $y=1$ as examples where $(x*y \neq y*x) \land (x+y \neq y+x)$.

  \item
    \begin{question}
      Show that the following sentences are invalid:
    \end{question}
    \begin{enumerate}
      \item 
        \begin{question}
          $$\forall x \exists y Q(x,y) \implies \exists x \forall y Q(x,y)$$
        \end{question}

        Let's use $\mathcal{Z}$ to be the interpretation of the symbols as integers, and define:
        $$Q(x,y) \iff x < y$$
        Then, the left side of the implication always holds, but the right side of the implication never holds, meaning that the sentence is invalid.
      \item 
        \begin{question}
          $$(\forall x Q(x,x) \land \forall x \forall y (Q(x,y) \implies Q(y,x))) \implies \forall x \forall y \forall z (Q(x,y) \land Q(y,z) \implies Q(x,z))$$
        \end{question}

        Let's use $\mathcal{N}$ to be the interpretation of the symbols as natural numbers, and define:
        $$Q(a,b) \iff (a=b) \lor (a+b < 100)$$

        Then, we just need to show that:
        $$(\forall x Q(x,x) \land \forall x \forall y (Q(x,y) \implies Q(y,x))) \land (\exists x \exists y \exists z (Q(x,y) \land Q(y,z) \land \neg Q(x,z)))$$

        The left side clearly holds for our defined function (by construction), and if we define $x=0, y=150, z=50$, then the right side is also true, since:
        $$Q(0,150) \land Q(150,50) \land \neg Q(0,50)$$

    \end{enumerate}

  \item 
    Show that:

    \begin{enumerate}
      \item
        \begin{question}
          If $\Gamma \cup \{\neg (B \land C)\}$ is satisfiable, then either $\Gamma \cup \{\neg B \}$ is satisfiable or $\Gamma \cup \{\neg C\}$ is satisfiable.
        \end{question}

        If $\Gamma \cup \{\neg (B \land C)\}$ is satisfiable, then we know that $\Gamma \cup \{\neg B \lor \neg C\}$ is satisfiable. This denotes adding either $\neg B$ or $\neg C$ to $\Gamma$. This means that either $\Gamma \cup \{\neg B \}$ is satisfiable or $\Gamma \cup \{\neg C\}$ is satisfiable (from (a) of the satisfiability principles from the textbook).

      \item
        \begin{question}
          If $\Gamma \cup \{\neg \forall x B(x) \}$ is satisfiable, then for any constant $c$ not occuring in $\Gamma$ or $\forall x B(x)$, $\Gamma \cup \{\neg B(c)\}$ is satisfiable.
        \end{question}

         If $\Gamma \cup \{\neg \forall x B(x) \}$ is satisfiable, then $\Gamma \cup \{\exists x \neg B(x) \}$ is satisfiable. This means that for any constant $c$ not occuring in $\Gamma$ or $\forall x B(x)$, $\Gamma \cup \{\neg B(c)\}$ is satisfiable (from (b) of the satisfiability principles).
    \end{enumerate}
\end{enumerate}

\end{document}