\documentclass[12t,letterpaper]{article}

\usepackage[framemethod=default]{mdframed}

\newenvironment{proof}{\noindent{\bf Proof:}}{\qed\bigskip}

\newtheorem{theorem}{Theorem}
\newtheorem{corollary}{Corollary}
\newtheorem{lemma}{Lemma} 
\newtheorem{claim}{Claim}
\newtheorem{fact}{Fact}
\newtheorem{definition}{Definition}
\newtheorem{assumption}{Assumption}
\newtheorem{observation}{Observation}
\newtheorem{example}{Example}
\newcommand{\qed}{\rule{7pt}{7pt}}

\newcommand{\assignment}[4]{
\thispagestyle{plain} 
\newpage
\setcounter{page}{1}
\noindent
\begin{center}
\framebox{ \vbox{ \hbox to 6.28in
{\bf Phil 140b: Intermediate Logic \hfill #1}
\vspace{4mm}
\hbox to 6.28in
{\hspace{2.5in}\large\mbox{#2}}
\vspace{4mm}
\hbox to 6.28in
{{\it Handed Out: #3 \hfill Due: #4}}
}}
\end{center}
}

\newcommand{\solution}[3]{
\thispagestyle{plain} 
\newpage
\setcounter{page}{1}
\noindent
\begin{center}
\framebox{ \vbox{ \hbox to 6.28in
{\bf Phil 140b \hfill #3}
\vspace{4mm}
\hbox to 6.28in
{\hspace{2.5in}\large\mbox{#2}}
\vspace{4mm}
\hbox to 6.28in
{#1 \hfill}
}}
\end{center}
\markright{#1}
}

\newcommand\tarski{\mathrel{|}\!=}

\newenvironment{algorithm}
{\begin{center}
\begin{tabular}{|l|}
\hline
\begin{minipage}{1in}
\begin{tabbing}
\quad\=\qquad\=\qquad\=\qquad\=\qquad\=\qquad\=\qquad\=\kill}
{\end{tabbing}
\end{minipage} \\
\hline
\end{tabular}
\end{center}}

\newenvironment{question}
{\begin{quote}\itshape}
{\end{quote}}

\def\Comment#1{\textsf{\textsl{$\langle\!\langle$#1\/$\rangle\!\rangle$}}}


\usepackage{amsmath, dsfont, mathtools, verbatim, tikz, float, textcomp, mathrsfs, amssymb, centernot}

\usetikzlibrary{arrows,automata}

\oddsidemargin 0in
\evensidemargin 0in
\textwidth 6.5in
\topmargin -0.5in
\textheight 9.0in
\newcommand{\norm}[1]{\left\lVert #1 \right\rVert}
\newcommand{\abs}[1]{\left\vert #1 \right\vert}
\newcommand{\?}{\stackrel{?}{=}}
\newcommand{\s}[1]{\overbrace{0''^{\cdots}}^{\text{#1 times}}}
\newcommand{\brak}[1]{\ulcorner #1 \urcorner}
\newcommand{\Diag}[1]{\text{Diag}(#1)}
\DeclarePairedDelimiter{\ceil}{\lceil}{\rceil}




\begin{document}

\solution{Nikhil Unni}{Assignment \#6}{Spring 2016}
\pagestyle{myheadings}

\begin{enumerate}
  \item
    \begin{question}
      Suppose that $B(y)$ is a provability predicate for $T$, where $T$ is consistent. Let $D(y)$ be the formula:
      $$B(y) \land y \neq \brak{0=1}$$
      Show that $D(y)$ meets the third condition of the definition of a provability predicate but not the second.
    \end{question}

    The third provability condition of a provability predicate $D(y)$ is:
    $$\tarski_T D(\brak{A}) \implies D(\brak{D(\brak{A})})$$
    $$\tarski_T B(\brak{A}) \land \brak{A} \neq \brak{0=1} \implies D(\brak{B(\brak{A}) \land \brak{A} \neq \brak{0=1}})$$
    $$\tarski_T B(\brak{A}) \land \brak{A} \neq \brak{0=1} \implies B(\brak{B(\brak{A}) \land \brak{A} \neq \brak{0=1}}) \land \brak{B(\brak{A}) \land \brak{A} \neq \brak{0=1}} \neq \brak{0=1}$$

    Because of (P1), we know that:
    $$\tarski_T B(\brak{A}) \land \brak{A} \neq \brak{0=1} \implies B(\brak{B(\brak{A}) \land \brak{A} \neq \brak{0=1}})$$

    And using the coding scheme discussed in class (and chapter 15), we know that since $\brak{0=1}$ is inside the sentence $B(\brak{A}) \land \brak{A} \neq \brak{0=1}$, we know that the number $\brak{B(\brak{A}) \land \brak{A} \neq \brak{0=1}}$ must be some number where there are a sequence of digits on either side of $\brak{0=1}$, so we know that $\brak{B(\brak{A}) \land \brak{A} \neq \brak{0=1}}$ can't be equal to $\brak{0=1}$. With this we have the third provability condition:
    $$\tarski_T B(\brak{A}) \land \brak{A} \neq \brak{0=1} \implies B(\brak{B(\brak{A}) \land \brak{A} \neq \brak{0=1}}) \land \brak{B(\brak{A}) \land \brak{A} \neq \brak{0=1}} \neq \brak{0=1}$$

    For the second provability condition of $D(y)$ to be incorrect would mean:
    $$\tarski_T D(\brak{A_1 \implies A_2}) \centernot \implies (D(\brak{A_1}) \implies D(\brak{A_2}))$$
    Replacing all of the negations:
    $$\tarski_T B(\brak{A_1 \implies A_2}) \land \brak{A_1 \implies A_2} \neq \brak{0=1} \centernot \implies (B(\brak{A_1}) \land \brak{A_1} \neq \brak{0=1} \implies B(\brak{A_2}) \land \brak{A_2} \neq \brak{0=1})$$
    $$\tarski_T B(\brak{A_1 \implies A_2}) \land \brak{A_1 \implies A_2} \neq \brak{0=1} \land \neg (B(\brak{A_1}) \land \brak{A_1} \neq \brak{0=1} \implies B(\brak{A_2}) \land \brak{A_2} \neq \brak{0=1})$$
    $$\tarski_T B(\brak{A_1 \implies A_2}) \land \brak{A_1 \implies A_2} \neq \brak{0=1} \land (B(\brak{A_1}) \land \brak{A_1} \neq \brak{0=1} \land \neg (B(\brak{A_2}) \land \brak{A_2} \neq \brak{0=1}))$$
    $$\tarski_T B(\brak{A_1 \implies A_2}) \land \brak{A_1 \implies A_2} \neq \brak{0=1} \land B(\brak{A_1}) \land \brak{A_1} \neq \brak{0=1} \land (\neg B(\brak{A_2}) \lor \brak{A_2} = \brak{0=1})$$
    As discussed above, it should be clear that $\brak{A_1 \implies A_2} \neq \brak{0=1}$ is true by the construction of codes. So instead we can show the truth of:
    $$\tarski_T B(\brak{A_1 \implies A_2}) \land B(\brak{A_1}) \land \brak{A_1} \neq \brak{0=1} \land (\neg B(\brak{A_2}) \lor \brak{A_2} = \brak{0=1})$$
    If we have a consistent system and if $A_1$ is not $0 = 1$, which would imply \textbf{any} sentence, from (P2) we have:
    $$\tarski_T B(\brak{A_1}) \land B(\brak{A_1 \implies A_2}) \implies B(\brak{A_2})$$

    However, if $B(\brak{A_1}) \land B(\brak{A_1 \implies A_2}) \land \neg B(\brak{A_2})$, we have an inconsistency with our last property, and so we can prove anything, namely $\brak{A_2} = \brak{0 = 1}$.
    So the statement \textbf{must} be true, meaning that the second provability predicate condition is false.
  \item
    \begin{question}
      Let $B(y)$ be a provability predicate for $T$ (extending $Q$). Show that $T$ proves the following:
    \end{question}

    \begin{enumerate}
      \item [(i)]
        \begin{question}
          $B(\brak{A \land C}) \iff (B(\brak{A}) \land B(\brak{C}))$
        \end{question}

        First we can show:
        $$B(\brak{\neg A}) \iff \neg B(\brak{A})$$
        We know that if you can prove $\neg A$, then there is no way to prove $A$, since this would introduce an inconsistency to the system. So we have:

        $$B(\brak{A \land C})$$
        $$\iff B(\brak{\neg (\neg A \lor \neg C)})$$
        $$\iff B(\brak{\neg (A \implies \neg C)})$$
        Using what we proved above:
        $$\iff \neg B(\brak{A \implies \neg C})$$
        And using (P2):
        $$\iff \neg (B(\brak{A}) \implies B(\brak{\neg C}))$$
        $$\iff \neg (B(\brak{A}) \implies \neg B(\brak{C}))$$
        $$\iff \neg (\neg B(\brak{A}) \lor \neg B(\brak{C}))$$
        $$\iff B(\brak{A}) \land B(\brak{C})$$
      \item [(ii)]
        \begin{question}
          $B(\brak{0=1}) \implies B(\brak{A})$
        \end{question}

        Since $Q$ has $0 \neq 1$ as an axiom, T is consistent if and only if $\centernot \tarski_T B(\brak{0=1})$. So if $\tarski_T B(\brak{0=1})$, then T is inconsistent, and so for any sentence $A$:
        $$\tarski_T A$$
        And from (P1), we have $\tarski_T B(\brak{A})$.
      \item [(iii)]
        \begin{question}
          $B(\brak{A}) \implies (B(\brak{\neg A}) \implies B(\brak{0=1}))$
        \end{question}

        This is equivalent to:
        $$\neg B(\brak{A}) \lor (\neg B(\brak{\neg A}) \lor B(\brak{0=1}))$$
        And using the negation property we showed in (i):
        $$\neg B(\brak{A}) \lor B(\brak{A}) \lor B(\brak{0=1})$$
        This is true simply because if we have a consistent system, then either $B(\brak{A})$ or $\neg B(\brak{A})$, and if we don't have a consistent system, then $B(\brak{0=1})$.
    \end{enumerate}


  \item
    \begin{question}
      Suppose $B(y)$ is a provability predicate for $T$. Use the existence of a sentence $G$ such that:
      $$\tarski_T G \iff B(\brak{G})$$
      to construct an ``alternative'' proof that if $T$ is consistent, then not:
      $$\tarski_T \neg B(\brak{0=1})$$
      Suggestion : show
      $$\tarski_T B(\brak{B(\brak{G})}) \implies B(\brak{\neg G}),$$
      $$\tarski_T B(\brak{G}) \implies B(\brak{\neg G}),$$ and
      $$\tarski_T B(\brak{G}) \implies (B(\brak{\neg G}) \implies B(\brak{0=1}))$$
      Conclude that if $\tarski_T \neg B(\brak{0=1})$, then:
      $$\tarski_T \neg B(\brak{G}),$$
      $$\tarski_T G,$$ and
      $$\tarski_T B(\brak{G})$$
    \end{question}

    Assume:
    $$(1) \tarski_T B(\brak{B(\brak{G})}) \implies B(\brak{\neg G})$$
    Then, from (P3), we have:
    $$(2) \tarski_T B(\brak{G}) \implies B(\brak{B(\brak{G})}) \implies B(\brak{\neg G})$$
    Since $T$ is consistent, $\neg B(\brak{G}) \lor \neg B(\brak{\neg G}) \lor B(\brak{0=1})$ must be true, since you cannot prove both $G$ and $\neg G$ while still maintaining a consistent system. This is equivalent to:
    $$(3) \tarski_T B(\brak{G}) \implies (B(\brak{\neg G}) \implies B(\brak{0=1}))$$

    With this all in mind, suppose that $\tarski_T \neg B(\brak{0=1})$. Then if $\tarski_T B(\brak{G})$, then from (3), we know that $\tarski_T \neg B(\brak{\neg G})$. And from (2), we know that $\tarski_T \neg B(\brak{G})$, which is an inconsistency. But if $\tarski_T \neg B(\brak{G})$, then we get $\tarski_T G$ and $\tarski_T B(\brak{G})$, which is also inconsistent.\\

    Since our assumption lead to an inconsistency, we know $\centernot \tarski_T \neg B(\brak{0=1})$


  \item
    \begin{question}
      Suppose that $B(y)$ is a provability predicate for $T$ and that $\tarski_T B(\brak{A}) \implies C$ and that $\tarski_T B(\brak{C}) \implies A$. Show that $\tarski_T A$ and $\tarski_T C$.
    \end{question}

    Let $D(y)$ be the formula $B(y) \implies A$, and let $E(y)$ be the formula $B(y) \implies C$ and apply the diagonal lemma to obtain sentences $D, E$ such that:
    $$(1) \tarski_T D \iff B(\brak{D}) \implies A$$
    $$(2) \tarski_T E \iff B(\brak{E}) \implies C$$
    By (P1), we have:
    $$(3) \tarski_T B(\brak{D \iff B(\brak{D}) \implies A})$$
    $$(4) \tarski_T B(\brak{E \iff B(\brak{E}) \implies C})$$

    By (P2), we have:
    $$(5) \tarski_T B(\brak{D}) \implies B(\brak{B(\brak{D}) \implies A})$$
    $$(6) \tarski_T B(\brak{E}) \implies B(\brak{B(\brak{E}) \implies C})$$
    $$(7) \tarski_T B(\brak{D}) \implies (B(\brak{B(\brak{D})}) \implies B(\brak{A}))$$
    $$(8) \tarski_T B(\brak{E}) \implies (B(\brak{B(\brak{E})}) \implies B(\brak{C}))$$
    From (P3) we have:
    $$(9) \tarski_T B(\brak{D}) \implies B(\brak{B(\brak{D})})$$
    $$(10) \tarski_T B(\brak{E}) \implies B(\brak{B(\brak{E})})$$
    Combining (7) and (9) and (8) and (10) we get:
    $$(11) \tarski_T B(\brak{D}) \implies B(\brak{A})$$
    $$(12) \tarski_T B(\brak{E}) \implies B(\brak{C})$$

    Combining (11) and (12) with the formulas stated in the problem:
    $$(13) \tarski_T B(\brak{D}) \implies C$$
    $$(14) \tarski_T B(\brak{E}) \implies A$$

    Then combining (1) with (13) and (2) with (14) we get:
    $$(15) \tarski_T D$$
    $$(16) \tarski_T E$$

    And with (P1):
    $$(17) \tarski_T B(\brak{D})$$
    $$(18) \tarski_T B(\brak{E})$$

    And combining (13) with (17) and (14) with (18) we finally get:
    $$\tarski_T C$$
    $$\tarski_T A$$

\end{enumerate}

\end{document}