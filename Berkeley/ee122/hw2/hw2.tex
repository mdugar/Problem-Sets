\documentclass[12t,letterpaper]{article}

\newenvironment{proof}{\noindent{\bf Proof:}}{\qed\bigskip}

\newtheorem{theorem}{Theorem}
\newtheorem{corollary}{Corollary}
\newtheorem{lemma}{Lemma} 
\newtheorem{claim}{Claim}
\newtheorem{fact}{Fact}
\newtheorem{definition}{Definition}
\newtheorem{assumption}{Assumption}
\newtheorem{observation}{Observation}
\newtheorem{example}{Example}
\newcommand{\qed}{\rule{7pt}{7pt}}

\newcommand{\assignment}[4]{
\thispagestyle{plain} 
\newpage
\setcounter{page}{1}
\noindent
\begin{center}
\framebox{ \vbox{ \hbox to 6.28in
{\bf EE 122: Communication Networks \hfill #1}
\vspace{4mm}
\hbox to 6.28in
{\hspace{2.5in}\large\mbox{#2}}
\vspace{4mm}
\hbox to 6.28in
{{\it Handed Out: #3 \hfill Due: #4}}
}}
\end{center}
}

\newcommand{\solution}[3]{
\thispagestyle{plain} 
\newpage
\setcounter{page}{1}
\noindent
\begin{center}
\framebox{ \vbox{ \hbox to 6.28in
{\bf EE 122 \hfill #3}
\vspace{4mm}
\hbox to 6.28in
{\hspace{2.5in}\large\mbox{#2}}
\vspace{4mm}
\hbox to 6.28in
{#1 \hfill}
}}
\end{center}
\markright{#1}
}

\newenvironment{algorithm}
{\begin{center}
\begin{tabular}{|l|}
\hline
\begin{minipage}{1in}
\begin{tabbing}
\quad\=\qquad\=\qquad\=\qquad\=\qquad\=\qquad\=\qquad\=\kill}
{\end{tabbing}
\end{minipage} \\
\hline
\end{tabular}
\end{center}}

\def\Comment#1{\textsf{\textsl{$\langle\!\langle$#1\/$\rangle\!\rangle$}}}


\usepackage{amsmath, dsfont, tikz, float}

\usetikzlibrary{arrows,automata,positioning}

\oddsidemargin 0in
\evensidemargin 0in
\textwidth 6.5in
\topmargin -0.5in
\textheight 9.0in

\newenvironment{amatrix}[1]{%
  \left(\begin{array}{@{}*{#1}{c}|c@{}}
}{%
  \end{array}\right)
}

\makeatletter
\renewcommand*\env@matrix[1][*\c@MaxMatrixCols c]{%
  \hskip -\arraycolsep
  \let\@ifnextchar\new@ifnextchar
  \array{#1}}
\makeatother

\newcommand{\norm}[1]{\left\lVert #1 \right\rVert}
\newcommand{\?}{\stackrel{?}{=}}
\newcommand\given[1][]{\:#1\vert\:}


\begin{document}

\solution{Nikhil Unni}{HW1}{Spring 2016}
\pagestyle{myheadings}

\begin{enumerate}
  \item
  \begin{enumerate}   
  \item [1.1] Just writing out the probabilities in order, it's simply:
    $$
      \begin{pmatrix}[ccc]
        0 & 1-p & p\\
        1 & 0   & 0\\
        0 & \alpha & 1-\alpha\\
      \end{pmatrix}
    $$

  \item [1.2] We have paths to all points from all points (it is irreducible) as long as both $p > 0$ and $\alpha > 0$. And it is also aperiodic under the same conditions.
  \item [1.3]
    $$
      \begin{pmatrix}[ccc]
        0 & 1-p & p\\
        1 & 0   & 0\\
        0 & \alpha & 1-\alpha\\
      \end{pmatrix}
      \begin{pmatrix}[c]
        x\\
        y\\
        z\\
      \end{pmatrix}
      = 
      \begin{pmatrix}[c]
        x\\
        y\\
        z\\
      \end{pmatrix}
    $$

    Separating it out as a system of equations we get:
    $$(1-p)y + pz = x$$
    $$x = y$$
    $$\alpha y + (1 - \alpha)z = z$$

    Solving it, we find that $x = y = z$. So:
    $$\pi =       
    \begin{pmatrix}[c]
        1/3\\
        1/3\\
        1/3\\
      \end{pmatrix}
    $$
  \end{enumerate}

  \item
    \begin{enumerate}
      \item [2.1] The states of the system are $\{0, 1, \cdots, k\}$.
      \item [2.2] 

    \end{enumerate}
  \item We can calculate this with Little's Result. First we need to find the average occupancy. The packets coming in on average have $\frac{4000+400}{2} = 2200$ bits. And the average Mbps is $0.8*4 + 0.2*1 = 3.4$ Mbps. So the average occupancy should be $2200 * \frac{1}{3.4} * \frac{1}{10^6} = .000647059$ seconds. So then, average delay is $\frac{.000647059}{100} = .000006471$ seconds, or $.006471$ ms.

  \item 
    \begin{enumerate}
      \item [4.1] It can be modeled with a CTMC because, if we're at any number of lines, call it $k$, with probability $\lambda$, we could get another call at any point in time, but with probability $\mu$, they could finish a call at any point in time. And because of the ``memoryless'' property of the Poisson process and the Exponential distribution, the amount of time we've already waited for a new call to come in, or for a call to finish does not change the probability of when the call is going to finish or when a new call is going to come in. Given all of these Markov properties, it can be modeled with a CTMC.

      \item [4.2] With $\lambda = \frac{30}{60} = 1/2$ call/min, and $\mu = 4$ min, the matrix would look like:
    $$Q =       
    \begin{pmatrix}[cccc]
        -1/2 & 1/2 & 0 & 0\\
        4 & -9/2 & 1/2 & 0\\
        0 & 4 & -9/2 & 1/2\\
        0 & 0 & 4 & -9/2\\
      \end{pmatrix}
    $$
      
      \item [4.3] We can show it by Little's Result.
      \item [4.4] From the lecture slides, we know $\pi(n) = (1-\rho)^n \rho^n$. So we get:
        $$\pi = 
        \begin{pmatrix}[c]
          1 - \rho\\
          \rho - \rho^2\\
          \rho^2 - \rho^3\\
          \rho^3 - \rho^4\\
        \end{pmatrix}
        =
        \begin{pmatrix}[c]
          7/8\\
          7/64\\
          7/512\\
          7/4096\\
        \end{pmatrix}
        $$
        And so the blocking probability is $\frac{7}{4096}$.

      \item [4.5] Looking at the invariant distribution, they should subscribe to at least two phone lines, since the blocking probability at just one phone line is slightly over $10\%$.
    \end{enumerate}

  \item 
    \begin{itemize}
      \item [5.1] Again, using the $\rho$ formula above, it's the probability of $\pi_1(2)\pi_2(3)$. This is just $(1-\frac{\lambda_1}{\mu_1})^n(\frac{\lambda_1}{\mu_1})^n * (1-\frac{\lambda_2}{\mu_2})^n(\frac{\lambda_2}{\mu_2})^n$, with $\lambda_1 = \lambda_2 = \frac{1}{6}$ customers per minute, $\mu_1 = 2$ minutes, and $\mu_2 = 4$ minutes.


      \item [5.2] Using Little's Law again, we know the average occupancy is $2+4 = 6$ minutes. And the arrival rate is $10$ customers per hour, or $\frac{1}{6}$ customers per minute. So then the average delay is $6*6 = 36$ minutes.

      \item [5.3] We can treat this problem as an M/M/1 of arrival rate $\frac{1}{6}$ customers per minute, and average service time of $6$ minutes. Then, $\pi(4) = (1-\frac{1}{36})^n(\frac{1}{36})^n = .000000532$
    \end{itemize}


  \item 
    \begin{itemize}
      \item [6.1] Yes, we could still possibly transmit high-fidelity radio. If we increase the power of the signal, we may be able to achieve a high enough SNR.

      \item [6.2] Under Shannon's Theorem: $C = Wlog_2(1 + \frac{S}{N})$. Plugging in, we get:
        $$270880 = 200000log_2(1 + SNR)$$
        $$SNR = 1.5569$$

        With an $SNR$ of $10$, we get:
        $$C = 200000log_2(1 + 10) = 691.88 \text{ Kbps}$$
    \end{itemize}
\end{enumerate}

\end{document}
