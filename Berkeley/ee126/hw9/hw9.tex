\documentclass[12t,letterpaper]{article}

\newenvironment{proof}{\noindent{\bf Proof:}}{\qed\bigskip}

\newtheorem{theorem}{Theorem}
\newtheorem{corollary}{Corollary}
\newtheorem{lemma}{Lemma} 
\newtheorem{claim}{Claim}
\newtheorem{fact}{Fact}
\newtheorem{definition}{Definition}
\newtheorem{assumption}{Assumption}
\newtheorem{observation}{Observation}
\newtheorem{example}{Example}
\newcommand{\qed}{\rule{7pt}{7pt}}

\newcommand{\assignment}[4]{
\thispagestyle{plain} 
\newpage
\setcounter{page}{1}
\noindent
\begin{center}
\framebox{ \vbox{ \hbox to 6.28in
{\bf EE 126: Probability and Random Processes \hfill #1}
\vspace{4mm}
\hbox to 6.28in
{\hspace{2.5in}\large\mbox{#2}}
\vspace{4mm}
\hbox to 6.28in
{{\it Handed Out: #3 \hfill Due: #4}}
}}
\end{center}
}

\newcommand{\solution}[3]{
\thispagestyle{plain} 
\newpage
\setcounter{page}{1}
\noindent
\begin{center}
\framebox{ \vbox{ \hbox to 6.28in
{\bf EE 126 \hfill #3}
\vspace{4mm}
\hbox to 6.28in
{\hspace{2.5in}\large\mbox{#2}}
\vspace{4mm}
\hbox to 6.28in
{#1 \hfill}
}}
\end{center}
\markright{#1}
}

\newenvironment{algorithm}
{\begin{center}
\begin{tabular}{|l|}
\hline
\begin{minipage}{1in}
\begin{tabbing}
\quad\=\qquad\=\qquad\=\qquad\=\qquad\=\qquad\=\qquad\=\kill}
{\end{tabbing}
\end{minipage} \\
\hline
\end{tabular}
\end{center}}

\def\Comment#1{\textsf{\textsl{$\langle\!\langle$#1\/$\rangle\!\rangle$}}}


\usepackage{amsmath, dsfont, mathtools, verbatim, tikz, float}

\usetikzlibrary{arrows,automata}

\oddsidemargin 0in
\evensidemargin 0in
\textwidth 6.5in
\topmargin -0.5in
\textheight 9.0in

\newenvironment{amatrix}[1]{%
  \left(\begin{array}{@{}*{#1}{c}|c@{}}
}{%
  \end{array}\right)
}

\DeclarePairedDelimiter{\ceil}{\lceil}{\rceil}
\DeclareMathOperator*{\argmin}{arg\,min}
\DeclareMathOperator*{\argmax}{arg\,max}

\makeatletter
\renewcommand*\env@matrix[1][*\c@MaxMatrixCols c]{%
  \hskip -\arraycolsep
  \let\@ifnextchar\new@ifnextchar
  \array{#1}}
\makeatother

\newcommand{\norm}[1]{\left\lVert #1 \right\rVert}
\newcommand{\abs}[1]{\left\vert #1 \right\vert}
\newcommand{\?}{\stackrel{?}{=}}
\newcommand\given[1][]{\:#1\vert\:}
\renewcommand{\d}[1]{\ensuremath{\operatorname{d}\!{#1}}}

\begin{document}

\solution{Nikhil Unni}{HW7}{Spring 2016}
\pagestyle{myheadings}

\begin{enumerate}
  \item 
  \item The random variable $X$ is exponentially distributed with mean $1$. Given $X$, the random variable $Y$ is exponentially distributed with rate $X$.
    \begin{enumerate}
      \item Find $MLE[X|Y]$.\\\\

        MLE should just be $\argmax_x P(X=x|Y=y) = \argmax_x P(Y=y|X=x)$, since all priors are equall likely. So we have:
        $$\argmax_x xe^{-xy}$$
        =$$\argmax_x \ln(x) - xy$$
        Taking the partial derivative we get:
        $$\frac{\partial}{\partial x} \ln(x) - xy = \frac{1}{x} - y$$
        Settng it to $0$ and solving for $x$, we have:
        $$\frac{1}{x} - y = 0 \implies x = \frac{1}{y}$$
      \item Find $MAP[X|Y]$.\\\\

        Again, we have $\argmax_x P(X=x|Y=y) = \argmax_x P(Y=y|X=x)P(X=x)$. Plugging in the actual distributions, we get:
        $$=\argmax_x e^{-x}(xe^{-xy})$$
        $$=\argmax_x \ln(x) - x(y+1)$$
        Taking the partial derivative and setting to $0$, we have:
        $$\frac{\partial}{\partial x} \ln(x) - x(y+1) = 0$$
        $$\frac{1}{x} - y - 1 = 0$$
        $$x = \frac{1}{y+1}$$
    \end{enumerate}

  \item The stochastic block model (SBM) as defined in Lab 9 is a random graph $G(n,p,q)$ consisting of two communities of size $\frac{n}{2}$ each such that the probability an edge exists between two nodes of the same community is $p$ and the probability an edge exists between two nodes in different communities is $q$, where $p > q$. The goal of the problem is to exactly determine the two communities given only the graph. Show that the MAP-decision rule is equivalent to finding the min-bisection of the graph.
  \item In this problem, we use similar settings which were considered in HW2. Consider a random bipartite graph, $G_1$, with $K$ left nodes, and $M$ right nodes. Each of the $KM$ possible edges of this graph is connected with probability $p$ independently. In the following problems, we consider the situations when $M$ and $K$ are large and $Mp$ and $Kp$ are constants. Hint : Use the Poisson distribution to approximate binomial distribution and apply law of large numbers.
    \begin{enumerate}
      \item A singleton is a right node of degree one. As $M$ and $K$ get large, how many left nodes are connected to right nodes which are singletons?
      \item A doubleton is a right node of degree two. As $M$ and $K$ get large, how many doubletons do we have?
      \item We call 2 doubletons distinct, if they are not connected to the same 2 left nodes. As $k$ and $M$ get large, what is the probability that two doubletons are distinct?
    \end{enumerate}
\end{enumerate}

\end{document}
